% ARTICLE 2 ----
% This is just here so I know exactly what I'm looking at in Rstudio when messing with stuff.
% Options for packages loaded elsewhere
\PassOptionsToPackage{unicode}{hyperref}
\PassOptionsToPackage{hyphens}{url}
%
\documentclass[
  11pt,
]{article}
\usepackage{lmodern}
\usepackage{amssymb,amsmath}
\usepackage{ifxetex,ifluatex}
\ifnum 0\ifxetex 1\fi\ifluatex 1\fi=0 % if pdftex
  \usepackage[T1]{fontenc}
  \usepackage[utf8]{inputenc}
  \usepackage{textcomp} % provide euro and other symbols
\else % if luatex or xetex
  \usepackage{unicode-math}
  \defaultfontfeatures{Scale=MatchLowercase}
  \defaultfontfeatures[\rmfamily]{Ligatures=TeX,Scale=1}
  \setmainfont[]{cochineal}
\fi
% Use upquote if available, for straight quotes in verbatim environments
\IfFileExists{upquote.sty}{\usepackage{upquote}}{}
\IfFileExists{microtype.sty}{% use microtype if available
  \usepackage[]{microtype}
  \UseMicrotypeSet[protrusion]{basicmath} % disable protrusion for tt fonts
}{}
\makeatletter
\@ifundefined{KOMAClassName}{% if non-KOMA class
  \IfFileExists{parskip.sty}{%
    \usepackage{parskip}
  }{% else
    \setlength{\parindent}{0pt}
    \setlength{\parskip}{6pt plus 2pt minus 1pt}
    }
}{% if KOMA class
  \KOMAoptions{parskip=half}}
\makeatother
\usepackage{xcolor}
\IfFileExists{xurl.sty}{\usepackage{xurl}}{} % add URL line breaks if available
\urlstyle{same} % disable monospaced font for URLs
\usepackage[margin=1in]{geometry}
\usepackage{color}
\usepackage{fancyvrb}
\newcommand{\VerbBar}{|}
\newcommand{\VERB}{\Verb[commandchars=\\\{\}]}
\DefineVerbatimEnvironment{Highlighting}{Verbatim}{commandchars=\\\{\}}
% Add ',fontsize=\small' for more characters per line
\usepackage{framed}
\definecolor{shadecolor}{RGB}{248,248,248}
\newenvironment{Shaded}{\begin{snugshade}}{\end{snugshade}}
\newcommand{\AlertTok}[1]{\textcolor[rgb]{0.94,0.16,0.16}{#1}}
\newcommand{\AnnotationTok}[1]{\textcolor[rgb]{0.56,0.35,0.01}{\textbf{\textit{#1}}}}
\newcommand{\AttributeTok}[1]{\textcolor[rgb]{0.77,0.63,0.00}{#1}}
\newcommand{\BaseNTok}[1]{\textcolor[rgb]{0.00,0.00,0.81}{#1}}
\newcommand{\BuiltInTok}[1]{#1}
\newcommand{\CharTok}[1]{\textcolor[rgb]{0.31,0.60,0.02}{#1}}
\newcommand{\CommentTok}[1]{\textcolor[rgb]{0.56,0.35,0.01}{\textit{#1}}}
\newcommand{\CommentVarTok}[1]{\textcolor[rgb]{0.56,0.35,0.01}{\textbf{\textit{#1}}}}
\newcommand{\ConstantTok}[1]{\textcolor[rgb]{0.00,0.00,0.00}{#1}}
\newcommand{\ControlFlowTok}[1]{\textcolor[rgb]{0.13,0.29,0.53}{\textbf{#1}}}
\newcommand{\DataTypeTok}[1]{\textcolor[rgb]{0.13,0.29,0.53}{#1}}
\newcommand{\DecValTok}[1]{\textcolor[rgb]{0.00,0.00,0.81}{#1}}
\newcommand{\DocumentationTok}[1]{\textcolor[rgb]{0.56,0.35,0.01}{\textbf{\textit{#1}}}}
\newcommand{\ErrorTok}[1]{\textcolor[rgb]{0.64,0.00,0.00}{\textbf{#1}}}
\newcommand{\ExtensionTok}[1]{#1}
\newcommand{\FloatTok}[1]{\textcolor[rgb]{0.00,0.00,0.81}{#1}}
\newcommand{\FunctionTok}[1]{\textcolor[rgb]{0.00,0.00,0.00}{#1}}
\newcommand{\ImportTok}[1]{#1}
\newcommand{\InformationTok}[1]{\textcolor[rgb]{0.56,0.35,0.01}{\textbf{\textit{#1}}}}
\newcommand{\KeywordTok}[1]{\textcolor[rgb]{0.13,0.29,0.53}{\textbf{#1}}}
\newcommand{\NormalTok}[1]{#1}
\newcommand{\OperatorTok}[1]{\textcolor[rgb]{0.81,0.36,0.00}{\textbf{#1}}}
\newcommand{\OtherTok}[1]{\textcolor[rgb]{0.56,0.35,0.01}{#1}}
\newcommand{\PreprocessorTok}[1]{\textcolor[rgb]{0.56,0.35,0.01}{\textit{#1}}}
\newcommand{\RegionMarkerTok}[1]{#1}
\newcommand{\SpecialCharTok}[1]{\textcolor[rgb]{0.00,0.00,0.00}{#1}}
\newcommand{\SpecialStringTok}[1]{\textcolor[rgb]{0.31,0.60,0.02}{#1}}
\newcommand{\StringTok}[1]{\textcolor[rgb]{0.31,0.60,0.02}{#1}}
\newcommand{\VariableTok}[1]{\textcolor[rgb]{0.00,0.00,0.00}{#1}}
\newcommand{\VerbatimStringTok}[1]{\textcolor[rgb]{0.31,0.60,0.02}{#1}}
\newcommand{\WarningTok}[1]{\textcolor[rgb]{0.56,0.35,0.01}{\textbf{\textit{#1}}}}
\usepackage{graphicx}
\makeatletter
\def\maxwidth{\ifdim\Gin@nat@width>\linewidth\linewidth\else\Gin@nat@width\fi}
\def\maxheight{\ifdim\Gin@nat@height>\textheight\textheight\else\Gin@nat@height\fi}
\makeatother
% Scale images if necessary, so that they will not overflow the page
% margins by default, and it is still possible to overwrite the defaults
% using explicit options in \includegraphics[width, height, ...]{}
\setkeys{Gin}{width=\maxwidth,height=\maxheight,keepaspectratio}
% Set default figure placement to htbp
\makeatletter
\def\fps@figure{htbp}
\makeatother
\setlength{\emergencystretch}{3em} % prevent overfull lines
\providecommand{\tightlist}{%
  \setlength{\itemsep}{0pt}\setlength{\parskip}{0pt}}
\setcounter{secnumdepth}{-\maxdimen} % remove section numbering

\ifluatex
  \usepackage{selnolig}  % disable illegal ligatures
\fi
\usepackage[]{natbib}
\bibliographystyle{apsr}


\title{How Influential are Music Critics?\thanks{This paper is prepared
for the course SURV727 taught by Ruben Bach and Christoph Kern jointly
at U Michigan and U Maryland. The project files can be found on my
\href{https://github.com/gnicholl/SURV727FinalProject}{github}. I use
the Article2 template developed by
\href{https://github.com/svmiller/stevetemplates}{Steve Miller} to
create this paper.}}
\author{true}
\date{December 16, 2021}

% Jesus, okay, everything above this comment is default Pandoc LaTeX template. -----
% ----------------------------------------------------------------------------------
% I think I had assumed beamer and LaTex were somehow different templates.


\usepackage{kantlipsum}

\usepackage{abstract}
\renewcommand{\abstractname}{}    % clear the title
\renewcommand{\absnamepos}{empty} % originally center

\renewenvironment{abstract}
 {{%
    \setlength{\leftmargin}{0mm}
    \setlength{\rightmargin}{\leftmargin}%
  }%
  \relax}
 {\endlist}

\makeatletter
\def\@maketitle{%
  \newpage
%  \null
%  \vskip 2em%
%  \begin{center}%
  \let \footnote \thanks
      {\fontsize{18}{20}\selectfont\raggedright  \setlength{\parindent}{0pt} \@title \par}
    }
%\fi
\makeatother


\title{How Influential are Music Critics?\thanks{This paper is prepared
for the course SURV727 taught by Ruben Bach and Christoph Kern jointly
at U Michigan and U Maryland. The project files can be found on my
\href{https://github.com/gnicholl/SURV727FinalProject}{github}. I use
the Article2 template developed by
\href{https://github.com/svmiller/stevetemplates}{Steve Miller} to
create this paper.}  }

\date{}

\usepackage{titlesec}

% 
\titleformat*{\section}{\large\bfseries}
\titleformat*{\subsection}{\normalsize\itshape} % \small\uppercase
\titleformat*{\subsubsection}{\normalsize\itshape}
\titleformat*{\paragraph}{\normalsize\itshape}
\titleformat*{\subparagraph}{\normalsize\itshape}

% add some other packages ----------

% \usepackage{multicol}
% This should regulate where figures float
% See: https://tex.stackexchange.com/questions/2275/keeping-tables-figures-close-to-where-they-are-mentioned
\usepackage[section]{placeins}



\makeatletter
\@ifpackageloaded{hyperref}{}{%
\ifxetex
  \PassOptionsToPackage{hyphens}{url}\usepackage[setpagesize=false, % page size defined by xetex
              unicode=false, % unicode breaks when used with xetex
              xetex]{hyperref}
\else
  \PassOptionsToPackage{hyphens}{url}\usepackage[draft,unicode=true]{hyperref}
\fi
}

\@ifpackageloaded{color}{
    \PassOptionsToPackage{usenames,dvipsnames}{color}
}{%
    \usepackage[usenames,dvipsnames]{color}
}
\makeatother
\hypersetup{breaklinks=true,
            bookmarks=true,
            pdfauthor={Gradon Nicholls (Non-Degree Student, Joint
Program in Survey Methodology, University of Maryland)},
             pdfkeywords = {Music Streaming, Experience Goods, R, APIs,
Web Scraping},
            pdftitle={How Influential are Music Critics?},
            colorlinks=true,
            citecolor=blue,
            urlcolor=blue,
            linkcolor=magenta,
            pdfborder={0 0 0}}
\urlstyle{same}  % don't use monospace font for urls

% Add an option for endnotes. -----



% This will better treat References as a section when using natbib
% https://tex.stackexchange.com/questions/49962/bibliography-title-fontsize-problem-with-bibtex-and-the-natbib-package
\renewcommand\bibsection{%
   \section*{References}%
   \markboth{\MakeUppercase{\refname}}{\MakeUppercase{\refname}}%
  }%

% set default figure placement to htbp
\makeatletter
\def\fps@figure{htbp}
\makeatother



\usepackage{longtable}
\LTcapwidth=.95\textwidth
\linespread{1.05}
\usepackage{hyperref}
\usepackage{float}
\floatplacement{figure}{H}
\usepackage{booktabs}
\usepackage{longtable}
\usepackage{array}
\usepackage{multirow}
\usepackage{wrapfig}
\usepackage{float}
\usepackage{colortbl}
\usepackage{pdflscape}
\usepackage{tabu}
\usepackage{threeparttable}
\usepackage{threeparttablex}
\usepackage[normalem]{ulem}
\usepackage{makecell}
\usepackage{xcolor}

\newtheorem{hypothesis}{Hypothesis}


% trick for moving figures to back of document
% really wish we'd knock this shit off with moving tables/figures to back of document
% but, alas...

% 
% Optional code chunks ------
% SOURCE: https://stackoverflow.com/questions/50702942/does-rmarkdown-allow-captions-and-references-for-code-chunks



\begin{document}

% \textsf{\textbf{This is sans-serif bold text.}}
% \textbf{\textsf{This is bold sans-serif text.}}


% \maketitle

{% \usefont{T1}{pnc}{m}{n}
\setlength{\parindent}{0pt}
\thispagestyle{plain}
{%\fontsize{18}{20}\selectfont\raggedright
\maketitle  % title \par

}




{
   \vskip 13.5pt\relax \normalsize\fontsize{11}{12}
   \MakeUppercase{Gradon Nicholls}, \small{Non-Degree Student, Joint
Program in Survey Methodology, University of Maryland}   

}

}








\begin{abstract}

%    \hbox{\vrule height .2pt width 39.14pc}

    \vskip 8.5pt % \small

\noindent \small{Put abstract here}


\vskip 8.5pt \noindent \emph{Keywords}: Music Streaming, Experience
Goods, R, APIs, Web Scraping \par

%    \hbox{\vrule height .2pt width 39.14pc}



\end{abstract}


\vskip -8.5pt


 % removetitleabstract


\setlength{\parindent}{16pt}
\setlength{\parskip}{0pt}

% We'll put doublespacing here
% Remember to cut it out later before bib
\hypertarget{introduction}{%
\section{Introduction}\label{introduction}}

When the quality of a product is unknown to the consumer prior to
consumption, the consumer must rely on the experiences of others in
order to make a purchase decision. These types of products are known as
``experience goods'' \citep{reinstein2005influence}, and include things
like restaurants, movies, TV shows, tourist destinations, and music.

Before purchasing experience goods, consumers may rely both on expert
reviews as well as reviews from other consumers or ``users''. Expert and
user reviews are not perfectly correlated and may provide differing
information: while expert reviews provide a sense of professionalism and
objectivity, user reviews (though biased towards highly negative or
highly positive scores) offer a relatable lay-person point of view
\citep{amblee2007freeware}. Use of review scores can be thought of as a
method for reducing search costs when researching a product
\citep{huang2009searching} in which case consumers must come up with
methods for integrating reviews from multiple sources
\citep{west1998integrating}, providing a demand for review aggregators
such as Rotten Tomatoes and Metacritic.

The goal of this paper is to study the role music critics play with
respect to the commercial performance of an artist's album. Anecdotally,
it seems to be the case that critics have an impact on the music
industry \citep{brennan2006rough}. For instance, the mere fact that
music critics exist at all, and have for so long, provides evidence that
their work serves a need in the market. Music labels would seem to
agree, as they often provide advance copies of their albums to reviewers
so that reviews can coincide with the album's release date.

The music industry, and with it the role of music critics, has evolved
as the economy becomes increasingly digitalized. Music can be consumed
rapidly and at very low cost. \citet{waldfogel201514} provides a summary
of the music industry in the age of ``new media,'' and notes the ability
of artists to find success despite a lack of airplay in traditional
media. An example is The Arctic Monkeys, whose early success is
attributed to the distribution of their demos by fans through the
internet \citep{morey2008arctic}--i.e.~they are one of the first bands
to ``go viral.''

In the age of new media, a music critic need not be a faceless author.
At the time of knitting this document, the music review channel
theneedledrop\footnote{\url{https://www.youtube.com/channel/UCt7fwAhXDy3oNFTAzF2o8Pw}}
has more than 2.5 million subscribers and nearly 800 million views
across all videos. Anthony Fantano, the curator of the channel, is not
just a reviewer but an internet personality with a dedicated following,
and has a side channel on YouTube for more ``off-the-cuff'' content.
Thus, I argue that Fantano may exist as a cross between an ``expert''
and ``user'' reviewer--in other words, he builds a lay-person trust
through his vlog-like content, while maintaining credibility as an
objective and thorough reviewer through his main channel.

In the new media landscape, we can hypothesize a clear causal pathway
between critic reviews and album performance that did not exist in a
pre-digital world. For example, in the description of a YouTube video or
in the text of an article, a critic can include a link which sends the
viewer or reader directly to listen to the album. Streaming platforms
like Spotify or Apple Music even allow you to embed their players in a
webpage for even more immediate listening. Artists have often complained
of low pay-out per stream on such platforms.\footnote{e.g.~\url{https://www.nytimes.com/2021/05/07/arts/music/streaming-music-payments.html}}
However, as of 2016, streaming accounted for a majority of revenue in
the industry, and growth in total revenue was driven mostly by paid
subscriptions to streaming services.\footnote{\url{http://www.riaa.com/wp-content/uploads/2017/03/RIAA-2016-Year-End-News-Notes.pdf}}
This, combined with the fact that streaming could also lead to fans
buying physical albums, merchandise, or concert tickets, make streaming
a relevant variable of interest, and it is therefore the focus of this
paper.

In order to correlate review scores with number of listens, I make use
of several data sources. I start with a list of top 1000 artists on
Spotify, and obtain information on their albums and release dates from
the Spotify api. For each album, I pull its respective article from the
Wikipedia api, and use web scraping techniques to obtain review scores
from the page (if they exist). Unfortunately, the Spotify api does not
provide the number of listens for an album. I therefore use the Last.fm
api to collect data on the number of ``scrobbles'', which is a term for
music listening data that Last.fm has collected from other apps that a
user is running.

In estimation, we should consider that a critic can take the role of
either an \emph{influencer} or a \emph{predictor}
\citep{eliashberg1997film}. As an \emph{influencer}, a positive review
could entice their audience to listen to an album. Additionally, the
review may act simply as a source of publicity \citep{sorensen2004any},
regardless of whether an album is scored positively or negatively. As a
\emph{predictor}, the reviewer may simply be forecasting the popularity
of an album. As discussed in \citet{reinstein2005influence}, this latter
effect is a spurious correlation, and does not represent the causal
effect that we are interested in.

I attempt to control for the predictor effect in two ways. First, I
assume that correlation between review scores and listens for older
albums represent the predictor effect only--that is, any effect critics
have had has dissipated over time. Under this reasoning, by including a
time dummy interacted with review scores, I should capture the
influencer effect at the margin. Second, I make use of artist fixed
effects to account for unobserved artist-specific variables. This should
control for predictor effects in certain cases--for example, if
reviewers tend to give good reviews to already-popular artists, or if
there are demographic effects.

Summary of results!

\hypertarget{data}{%
\section{Data}\label{data}}

I make use of several data sources in order to collect data on artists,
their albums, the number of listens for each album, and the review
scores of each album. As a high-level summary, the procedure is as
follows:

\begin{itemize}
\tightlist
\item
  Start with a list of artist names
\item
  For each artist name, use Spotify api to to find their albums
\item
  For each album, use Last.fm api to find the album's scrobbles
\item
  For each album, use the Google Custom Search api to find the url of
  the album's wikipedia page
\item
  Use web scraping techniques to obtain review scores from the wikipedia
  page
\end{itemize}

The following subsections provide details and code for each task. The
code to produce the full dataset takes a long time to run, so for
brevity I show an example of one artist and one album here. The full
code can be found in CreateData.R in the GitHub repository for this
project.

Note that we make use of the following packages:

\begin{Shaded}
\begin{Highlighting}[]
\ControlFlowTok{if}\NormalTok{ (}\SpecialCharTok{!}\FunctionTok{require}\NormalTok{(}\StringTok{"pacman"}\NormalTok{)) }\FunctionTok{install.packages}\NormalTok{(}\StringTok{"pacman"}\NormalTok{)}
\FunctionTok{library}\NormalTok{(pacman)}
\FunctionTok{p\_load}\NormalTok{(}\AttributeTok{char=}\FunctionTok{c}\NormalTok{(}\StringTok{"bookdown"}\NormalTok{,}\StringTok{"curl"}\NormalTok{,}\StringTok{"httr"}\NormalTok{,}\StringTok{"jsonlite"}\NormalTok{,}\StringTok{"rvest"}\NormalTok{,}\StringTok{"dplyr"}
\NormalTok{              ,}\StringTok{"papeR"}\NormalTok{,}\StringTok{"vtable"}\NormalTok{,}\StringTok{"stargazer"}\NormalTok{,}\StringTok{"sandwhich"}
\NormalTok{             ,}\StringTok{"spotifyr"}\NormalTok{,}\StringTok{"stringr"}\NormalTok{,}\StringTok{"tm"}\NormalTok{,}\StringTok{"utils"}\NormalTok{,}\StringTok{"WikipediR"}\NormalTok{,}\StringTok{"ggplot2"}\NormalTok{,}\StringTok{"plm"}\NormalTok{))}
\end{Highlighting}
\end{Shaded}

\hypertarget{list-of-artist-names}{%
\subsection{List of artist names}\label{list-of-artist-names}}

Given the uncountable number of artists that have ever existed, it is
not immediately clear how best to choose which artists should be
included in the analysis. For now, I prioritize convenience, and use a
list of the top 1000 Spotify artists as pulled by chartmasters.org. It
seems reasonable to start with the most popular artists given that they
make up the largest share of the market. However, it could bias results
if, for example, reviews have a greater impact on lesser-known artists.
To get a sense of the popularity of the artists, the number 1 and number
1000 artists currently have about 55 and 7 million monthly listeners,
respectively, at the time of writing.

Below is code\footnote{I make use of the
  \href{https://selectorgadget.com/}{selector gadget} to help with web
  scraping.} for scraping the list from chartmaster's webpage. Note that
the list is updated regularly, and so may differ slightly from my list
(due to a glitch, I collected the data in two batches, once on November
1 2021 and one on December 15, 2021). Currently, the top 4 artists on
Spotify are Drake, Ed Sheeran, Bad Bunny, and Ariana Grande.

\begin{Shaded}
\begin{Highlighting}[]
\NormalTok{chartmasterurl }\OtherTok{=} \StringTok{"https://chartmasters.org/most{-}streamed{-}artists{-}ever{-}on{-}spotify/"}
\NormalTok{chartwebpage }\OtherTok{=} \FunctionTok{read\_html}\NormalTok{(chartmasterurl)}
\NormalTok{xpath}\OtherTok{=}\StringTok{\textquotesingle{}//td[(((count(preceding{-}sibling::*) + 1) = 2) and parent::*)]\textquotesingle{}}
\NormalTok{artistnames }\OtherTok{=} \FunctionTok{html\_nodes}\NormalTok{(chartwebpage, }\AttributeTok{xpath=}\NormalTok{xpath)}
\NormalTok{artistnames }\OtherTok{=} \FunctionTok{html\_text}\NormalTok{(artistnames,}\AttributeTok{trim=}\NormalTok{T)}
\FunctionTok{head}\NormalTok{(artistnames,}\DecValTok{4}\NormalTok{)}
\end{Highlighting}
\end{Shaded}

\begin{verbatim}
## [1] "Drake"         "Ed Sheeran"    "Bad Bunny"     "Ariana Grande"
\end{verbatim}

\hypertarget{spotify-api}{%
\subsection{Spotify API}\label{spotify-api}}

To obtain album information for each artist, I make use of the Spotify
API. First, I search for the artist's name on Spotify and take the first
result. As long as the list of artists we start with is relatively close
to their official names, we should succeed in finding the correct
listing in Spotify. It further helps that in this case, the list I start
with is itself pulled from Spotify by chartmasters.org. Below, I obtain
the id number and artist name (as defined by Spotify) for Drake. In this
case, the artist's name is the same as what we started with.

\begin{Shaded}
\begin{Highlighting}[]
\NormalTok{search }\OtherTok{=}\NormalTok{ spotifyr}\SpecialCharTok{::}\FunctionTok{search\_spotify}\NormalTok{(}\AttributeTok{q=}\StringTok{"Drake"}\NormalTok{, }\AttributeTok{type=}\StringTok{"artist"}\NormalTok{)}
\NormalTok{artist\_id }\OtherTok{=}\NormalTok{ search}\SpecialCharTok{$}\NormalTok{id[}\DecValTok{1}\NormalTok{]}
\NormalTok{artist }\OtherTok{=}\NormalTok{ spotifyr}\SpecialCharTok{::}\FunctionTok{get\_artist}\NormalTok{(}\AttributeTok{id=}\NormalTok{artist\_id, }\AttributeTok{authorization=}\NormalTok{access\_token)}
\NormalTok{artist\_name }\OtherTok{=}\NormalTok{ artist}\SpecialCharTok{$}\NormalTok{name}
\NormalTok{artist\_name}
\end{Highlighting}
\end{Shaded}

\begin{verbatim}
## [1] "Drake"
\end{verbatim}

Using the artist id number, I obtain a list of Drake's albums in
Spotify. Note that albums can be pulled only 50 at a time. This is
accounted for in the main code by pulling albums 50 at a time until all
albums are found.

\begin{Shaded}
\begin{Highlighting}[]
\NormalTok{albums }\OtherTok{=}\NormalTok{ spotifyr}\SpecialCharTok{::}\FunctionTok{get\_artist\_albums}\NormalTok{(}\AttributeTok{id=}\NormalTok{artist\_id,}\AttributeTok{include\_groups=}\StringTok{"album"}
\NormalTok{                                    ,}\AttributeTok{market=}\StringTok{"CA"}\NormalTok{,}\AttributeTok{limit=}\DecValTok{50}
\NormalTok{                                    ,}\AttributeTok{authorization=}\NormalTok{access\_token)}
\NormalTok{albums }\SpecialCharTok{\%\textgreater{}\%} 
  \FunctionTok{select}\NormalTok{(}\FunctionTok{c}\NormalTok{(name,release\_date,total\_tracks)) }\SpecialCharTok{\%\textgreater{}\%}
  \FunctionTok{slice}\NormalTok{(}\DecValTok{16}\SpecialCharTok{:}\DecValTok{24}\NormalTok{)}
\end{Highlighting}
\end{Shaded}

\begin{verbatim}
##                                   name release_date total_tracks
## 1 If You're Reading This It's Too Late   2015-02-12           17
## 2 If You're Reading This It's Too Late   2015-02-12           17
## 3        Nothing Was The Same (Deluxe)   2013-01-01           16
## 4        Nothing Was The Same (Deluxe)   2013-01-01           15
## 5        Nothing Was The Same (Deluxe)   2013-01-01           16
## 6        Nothing Was The Same (Deluxe)   2013-01-01           16
## 7        Nothing Was The Same (Deluxe)   2013-01-01           15
## 8                 Nothing Was The Same   2013-01-01           13
## 9                 Nothing Was The Same   2013-01-01           13
\end{verbatim}

You will note that multiple versions of the same album exist in Spotify.
Sometimes they are identical, though other times they may have deluxe
versions, differing numbers of tracks, or differing release dates. Since
we do not pull listens from Spotify itself, we can remove duplicates
from the list without having to worry about aggregating albums together.
To remove duplicates, I first remove text contained within round or
square brackets. Next, I remove any albums that contain duplicate names
or release dates. In this case we are left with 12 distinct albums
released by Drake, as seen below:

\begin{Shaded}
\begin{Highlighting}[]
\NormalTok{albums}\SpecialCharTok{$}\NormalTok{name }\OtherTok{=} \FunctionTok{gsub}\NormalTok{(}\StringTok{"}\SpecialCharTok{\textbackslash{}\textbackslash{}}\StringTok{s*}\SpecialCharTok{\textbackslash{}\textbackslash{}}\StringTok{([\^{}}\SpecialCharTok{\textbackslash{}\textbackslash{}}\StringTok{)]+}\SpecialCharTok{\textbackslash{}\textbackslash{}}\StringTok{)"}\NormalTok{,}\StringTok{""}\NormalTok{,albums}\SpecialCharTok{$}\NormalTok{name)}
\NormalTok{albums }\OtherTok{=}\NormalTok{ albums[}\SpecialCharTok{!}\FunctionTok{duplicated}\NormalTok{(albums}\SpecialCharTok{$}\NormalTok{name),]}
\NormalTok{albums }\OtherTok{=}\NormalTok{ albums[}\SpecialCharTok{!}\FunctionTok{duplicated}\NormalTok{(albums}\SpecialCharTok{$}\NormalTok{release\_date),]}

\NormalTok{albums }\SpecialCharTok{\%\textgreater{}\%} 
  \FunctionTok{select}\NormalTok{(}\FunctionTok{c}\NormalTok{(name,release\_date,total\_tracks))}
\end{Highlighting}
\end{Shaded}

\begin{verbatim}
##                                    name release_date total_tracks
## 1                   Certified Lover Boy   2021-09-03           21
## 3                  Dark Lane Demo Tapes   2020-05-01           14
## 5                          Care Package   2019-08-02           17
## 7                           So Far Gone   2019-02-14           18
## 8                              Scorpion   2018-06-29           25
## 10                            More Life   2017-03-18           22
## 12                                Views   2016-05-06           20
## 14              What A Time To Be Alive   2015-09-25           11
## 16 If You're Reading This It's Too Late   2015-02-12           17
## 18                 Nothing Was The Same   2013-01-01           16
## 25                            Take Care   2011-11-15           19
## 31                       Thank Me Later   2010-01-01           15
\end{verbatim}

We can obtain information about an album's tracks by using the album's
id number provided by the API. Due to time constraints, I have not yet
experimented with using track analysis in this paper, and so all we need
for now is the album's name. For the remainder of this section we will
use Drake's most recent album, Certified Lover Boy, as the example.

\begin{Shaded}
\begin{Highlighting}[]
\NormalTok{album\_info}\OtherTok{=}\NormalTok{spotifyr}\SpecialCharTok{::}\FunctionTok{get\_album}\NormalTok{(}\AttributeTok{id=}\NormalTok{albums}\SpecialCharTok{$}\NormalTok{id[}\DecValTok{1}\NormalTok{]}
\NormalTok{                                  ,}\AttributeTok{authorization=}\NormalTok{access\_token)}
\NormalTok{album\_name }\OtherTok{=}\NormalTok{ album\_info}\SpecialCharTok{$}\NormalTok{name}
\NormalTok{album\_name}
\end{Highlighting}
\end{Shaded}

\begin{verbatim}
## [1] "Certified Lover Boy"
\end{verbatim}

\hypertarget{last.fm}{%
\subsection{Last.fm}\label{last.fm}}

As previously noted, Spotify does not provide access to listening
statistics through their API. I therefore turn to Last.fm to collect a
proxy measure. As of 2014, Last.fm does not offer its own streaming
service, but instead provides integration with Spotify and
YouTube.\footnote{\url{https://blog.last.fm/2014/01/29/did-someone-say-on-demand}}
Last.fm's main service is to allow users to track their music listening
behaviour across multiple devices and apps. To do this, it collects
information from other apps in the background, a process called
``scrobbling.''

The benefit of using Last.fm scrobbles in our analysis is that we obtain
data on listening behaviour from many different sources (for example,
both Spotify and Apple Music would be captured). The major downside is
that Last.fm is an opt-in service. This means there is a potential for
selection bias in our analysis. I hypothesize that Last.fm users
represent more dedicated music listeners, and thus may be more aware of
music news and reviews. If that is the case, correlations between
listening and review scores may be biased upwards, though are still
informative about a large subgroup of listeners.

That aside, obtaining scrobbles is fairly straightforward thanks to
Last.fm's API. Since Last.FM's data is scrobbled from other sources, the
official artist and album names from Spotify can be used to construct a
valid url in most cases, as long as we take care to strip out characters
that have special meanings.

\begin{Shaded}
\begin{Highlighting}[]
\NormalTok{url\_artist }\OtherTok{=} \FunctionTok{str\_replace\_all}\NormalTok{(artist\_name,}\StringTok{"}\SpecialCharTok{\textbackslash{}\textbackslash{}}\StringTok{+"}\NormalTok{,}\StringTok{"\%2B"}\NormalTok{)}
\NormalTok{url\_album  }\OtherTok{=} \FunctionTok{str\_replace\_all}\NormalTok{(album\_name,}\StringTok{"}\SpecialCharTok{\textbackslash{}\textbackslash{}}\StringTok{+"}\NormalTok{,}\StringTok{"\%2B"}\NormalTok{)}
\NormalTok{url\_artist }\OtherTok{=} \FunctionTok{str\_replace\_all}\NormalTok{(url\_artist,}\StringTok{"\&"}\NormalTok{,}\StringTok{"\%26"}\NormalTok{)}
\NormalTok{url\_album  }\OtherTok{=} \FunctionTok{str\_replace\_all}\NormalTok{(url\_album,}\StringTok{"\&"}\NormalTok{,}\StringTok{"\%26"}\NormalTok{)}
\NormalTok{url\_artist }\OtherTok{=} \FunctionTok{str\_replace\_all}\NormalTok{(url\_artist,}\StringTok{"\#"}\NormalTok{,}\StringTok{"\%23"}\NormalTok{)}
\NormalTok{url\_album  }\OtherTok{=} \FunctionTok{str\_replace\_all}\NormalTok{(url\_album,}\StringTok{"\#"}\NormalTok{,}\StringTok{"\%23"}\NormalTok{)}

\NormalTok{url }\OtherTok{=} \FunctionTok{paste0}\NormalTok{(}\StringTok{"http://ws.audioscrobbler.com/2.0/"}\NormalTok{,}
             \StringTok{"?method=album.getinfo"}\NormalTok{,}
             \StringTok{"\&api\_key="}\NormalTok{,}\StringTok{"xxxxxxxx"}\NormalTok{,}
             \StringTok{"\&artist="}\NormalTok{, }\FunctionTok{gsub}\NormalTok{(}\StringTok{" "}\NormalTok{, }\StringTok{"+"}\NormalTok{, url\_artist),}
             \StringTok{"\&album="}\NormalTok{, }\FunctionTok{gsub}\NormalTok{(}\StringTok{" "}\NormalTok{, }\StringTok{"+"}\NormalTok{, url\_album),}
             \StringTok{"\&format=json"}\NormalTok{)}
\NormalTok{url}
\end{Highlighting}
\end{Shaded}

\begin{verbatim}
## [1] "http://ws.audioscrobbler.com/2.0/?method=album.getinfo&api_key=xxxxxxxx&artist=Drake&album=Certified+Lover+Boy&format=json"
\end{verbatim}

Below we parse the JSON file in order to obtain the number of scrobbles.
A quick \href{https://www.last.fm/music/Drake/Certified+Lover+Boy}{check
on Last.FM} shows that the number we pulled is the correct one.

\begin{Shaded}
\begin{Highlighting}[]
\NormalTok{data\_json }\OtherTok{=}\NormalTok{ httr}\SpecialCharTok{::}\FunctionTok{GET}\NormalTok{(url)}
\NormalTok{data\_json }\OtherTok{=}\NormalTok{ jsonlite}\SpecialCharTok{::}\FunctionTok{fromJSON}\NormalTok{(}\FunctionTok{rawToChar}\NormalTok{(data\_json}\SpecialCharTok{$}\NormalTok{content))}
\NormalTok{album\_scrobbles }\OtherTok{=} \FunctionTok{as.integer}\NormalTok{(data\_json}\SpecialCharTok{$}\NormalTok{album}\SpecialCharTok{$}\NormalTok{playcount)}
\NormalTok{album\_scrobbles}
\end{Highlighting}
\end{Shaded}

\begin{verbatim}
## [1] 17603629
\end{verbatim}

\hypertarget{wikipedia}{%
\subsection{Wikipedia}\label{wikipedia}}

Metacritic is an obvious source for aggregated album review scores.
However, web scraping at a mass scale is not permitted on the
platform,\footnote{That being said, I place code for scraping individual
  Metacritic pages in the Appendix I.} and they do not offer an API
service. I therefore turn to Wikipedia for review score information.

Unfortunately, Wikipedia is not as exhaustive as Metacritic. In our
example of Drake's Certified Lover Boy, the
\href{https://www.metacritic.com/music/certified-lover-boy/drake}{Metacritic
page shows 20 reviews} while the corresponding
\href{https://en.wikipedia.org/wiki/Certified_Lover_Boy}{wiki page shows
only 10}. As will be seen later, most Wikipedia pages I collected
contained 10 or fewer reviews, suggesting that 10 may be the upper bound
chosen by Wikipedia moderators. It is worth further investigation as to
how reviews are chosen to be included. Interestingly, neither Wikipedia
nor Metacritic host scores from YouTube channel theneedledrop, meaning
that results may reflect old media more than new media.

Rather than try to construct the correct Wikipedia url by hand, I use
the Google Custom Search API, programmed to query only wikipedia.org, to
search the phrase ``{[}artist name{]} {[}album name{]} album''. The
output is a list of the 10 most relevant web pages as selected by
Google. I try the first result, and if that fails, assume review scores
and/or the desired article do not exist. Anecdotally this has worked
well, though more work should be done to test the accuracy of the
algorithm.

\begin{Shaded}
\begin{Highlighting}[]
\NormalTok{keyword }\OtherTok{=} \FunctionTok{paste0}\NormalTok{(artist\_name,}\StringTok{" "}\NormalTok{,album\_name,}\StringTok{" "}\NormalTok{,}\StringTok{"album"}\NormalTok{)}
\NormalTok{keyword }\OtherTok{=} \FunctionTok{gsub}\NormalTok{(}\StringTok{"[[:punct:]]"}\NormalTok{, }\StringTok{" "}\NormalTok{, keyword)}
\NormalTok{keyword }\OtherTok{=} \FunctionTok{str\_squish}\NormalTok{(keyword)}
\NormalTok{keyword }\OtherTok{=} \FunctionTok{gsub}\NormalTok{(}\StringTok{" "}\NormalTok{, }\StringTok{"+"}\NormalTok{, keyword)}

\NormalTok{url }\OtherTok{=} \FunctionTok{paste0}\NormalTok{(}\StringTok{"https://www.googleapis.com/customsearch/v1?"}
\NormalTok{             , }\StringTok{"key="}\NormalTok{, google.key}
\NormalTok{             , }\StringTok{"\&q="}\NormalTok{, keyword}
\NormalTok{             , }\StringTok{"\&gl=us"}         
\NormalTok{             , }\StringTok{"\&hl=en"}                
\NormalTok{             , }\StringTok{"\&cx="}\NormalTok{, google.cx}
\NormalTok{             , }\StringTok{"\&fields=items(link)"}
\NormalTok{)}
\NormalTok{googlesearch }\OtherTok{=} \FunctionTok{GET}\NormalTok{(url)}
\NormalTok{content }\OtherTok{=} \FunctionTok{rawToChar}\NormalTok{(googlesearch}\SpecialCharTok{$}\NormalTok{content)}
\NormalTok{searchresults }\OtherTok{=} \FunctionTok{fromJSON}\NormalTok{(content)}
\NormalTok{searchresults}
\end{Highlighting}
\end{Shaded}

\begin{verbatim}
## $items
##                                                                             link
## 1                              https://en.wikipedia.org/wiki/Certified_Lover_Boy
## 2                              https://de.wikipedia.org/wiki/Certified_Lover_Boy
## 3                                 https://en.wikipedia.org/wiki/Drake_(musician)
## 4                              https://fr.wikipedia.org/wiki/Certified_Lover_Boy
## 5                              https://es.wikipedia.org/wiki/Certified_Lover_Boy
## 6                                https://en.wikipedia.org/wiki/Drake_discography
## 7  https://en.wikipedia.org/wiki/List_of_Billboard_200_number-one_albums_of_2021
## 8                                 https://en.wikipedia.org/wiki/Girls_Want_Girls
## 9                https://en.wikipedia.org/wiki/Solid_(Young_Thug_and_Gunna_song)
## 10                                    https://en.wikipedia.org/wiki/In_the_Bible
\end{verbatim}

Rather than scrape the Wikipedia page, I make use of the Wikipedia API
to retrieve the page's content.

\begin{Shaded}
\begin{Highlighting}[]
\NormalTok{wikiurl }\OtherTok{=}\NormalTok{ searchresults}\SpecialCharTok{$}\NormalTok{items}\SpecialCharTok{$}\NormalTok{link[}\DecValTok{1}\NormalTok{]}

\CommentTok{\# deal with special characters}
\NormalTok{pgname }\OtherTok{=} \FunctionTok{curl\_unescape}\NormalTok{(wikiurl)}

\CommentTok{\# keep the end of the url {-}\textgreater{} the page name wikipedia api wants}
\NormalTok{pgname }\OtherTok{=} \FunctionTok{gsub}\NormalTok{(}\StringTok{"https://.*}\SpecialCharTok{\textbackslash{}\textbackslash{}}\StringTok{.wikipedia}\SpecialCharTok{\textbackslash{}\textbackslash{}}\StringTok{.org.*/"}\NormalTok{,}\StringTok{""}\NormalTok{,pgname)}

\CommentTok{\# obtain page content}
\NormalTok{webpage }\OtherTok{=}\NormalTok{ WikipediR}\SpecialCharTok{::}\FunctionTok{page\_content}\NormalTok{(}\AttributeTok{language=}\StringTok{"en"}\NormalTok{,}\AttributeTok{project=}\StringTok{"wikipedia"}\NormalTok{,}\AttributeTok{page\_name=}\NormalTok{pgname)}
\end{Highlighting}
\end{Shaded}

From there, there are quite a number of processing steps required in
order to properly format review scores. The complete code can be found
in Appendix II. In short, we must convert scores from various units to a
consistent scale (e.g.~3/5 stars, 8/10 stars, 60\%, 50/100). In the case
of letter grades (e.g.~B+), I use Metaritic's rules for converting to a
numeric grade\footnote{\url{https://www.metacritic.com/about-metascores}}.
Reviews based solely on words (e.g.~``positive'') are excluded from
analysis. I also pull release date information as a way to
cross-reference with Spotify.

The end result can be seen below:

\begin{Shaded}
\begin{Highlighting}[]
\NormalTok{release\_date\_wiki}
\end{Highlighting}
\end{Shaded}

\begin{verbatim}
## [1] "2021-09-03"
\end{verbatim}

\begin{Shaded}
\begin{Highlighting}[]
\NormalTok{wikiscores\_data}
\end{Highlighting}
\end{Shaded}

\begin{verbatim}
##                      V1 V2
## 3              AllMusic 60
## 4         The Arts Desk 60
## 5                 Clash 60
## 6  Entertainment Weekly 50
## 7          The Guardian 60
## 8       The Independent 40
## 9  The Line of Best Fit 50
## 10                  NME 40
## 11            Pitchfork 66
## 12        Rolling Stone 70
\end{verbatim}

\hypertarget{exploratory-data-analysis-and-cleaning}{%
\subsection{Exploratory Data Analysis and
Cleaning}\label{exploratory-data-analysis-and-cleaning}}

The result of the previous steps is a data frame with 12,694
albums.\footnote{For the full code, see Appendix III.} Here I load the
data. As previously mentioned, I pulled data in two batches, and so
append the two batches together, creating a flag for each batch so I can
account for the time discrepancy in estimation.

\begin{Shaded}
\begin{Highlighting}[]
\FunctionTok{load}\NormalTok{(}\AttributeTok{file=}\StringTok{"finaldata.RData"}\NormalTok{)}
\FunctionTok{load}\NormalTok{(}\AttributeTok{file=}\StringTok{"finaldata\_pt2.RData"}\NormalTok{)}

\NormalTok{finaldata }\OtherTok{=}\NormalTok{ finaldata }\SpecialCharTok{\%\textgreater{}\%}
  \FunctionTok{mutate}\NormalTok{(}\AttributeTok{batch=}\DecValTok{1}\NormalTok{) }\SpecialCharTok{\%\textgreater{}\%}
\NormalTok{  dplyr}\SpecialCharTok{::}\FunctionTok{select}\NormalTok{(}\SpecialCharTok{{-}}\NormalTok{isbad)}
\NormalTok{finaldata\_pt2 }\OtherTok{=}\NormalTok{ finaldata\_pt2 }\SpecialCharTok{\%\textgreater{}\%} \FunctionTok{mutate}\NormalTok{(}\AttributeTok{batch=}\DecValTok{2}\NormalTok{)}
\NormalTok{finaldata }\OtherTok{=} \FunctionTok{rbind}\NormalTok{(finaldata,finaldata\_pt2)}
\end{Highlighting}
\end{Shaded}

We need a method of aggregating review scores in order to conduct
analysis. Here I consider the number of available scores and the mean
score, but one could consider other measures like order statistics or
measures of spread.

\begin{Shaded}
\begin{Highlighting}[]
\NormalTok{finaldata}\SpecialCharTok{$}\NormalTok{numscores }\OtherTok{=}
  \FunctionTok{unlist}\NormalTok{(}\FunctionTok{lapply}\NormalTok{(finaldata}\SpecialCharTok{$}\NormalTok{wiki\_scores, }\AttributeTok{FUN =} \ControlFlowTok{function}\NormalTok{(y) \{}\FunctionTok{length}\NormalTok{(y)\}))}
\NormalTok{finaldata}\SpecialCharTok{$}\NormalTok{meanscores }\OtherTok{=}
  \FunctionTok{unlist}\NormalTok{(}\FunctionTok{lapply}\NormalTok{(finaldata}\SpecialCharTok{$}\NormalTok{wiki\_scores, }\AttributeTok{FUN =} \ControlFlowTok{function}\NormalTok{(y) \{}\FunctionTok{mean}\NormalTok{(y,}\AttributeTok{na.rm=}\NormalTok{T)\}))}
\end{Highlighting}
\end{Shaded}

First, let's look at the distribution of number of scores by album:

\begin{figure}
\centering
\includegraphics{__PAPER___files/figure-latex/unnamed-chunk-18-1.pdf}
\caption{Distribution of Number of Review Scores}
\end{figure}

We note first that there is a suspicious cluster at 10 reviews. As noted
above, Wikipedia pages are not ehaustive in their listing of review
scores, and it seems that 10 reviews is the usual maximum provided on
Wikipedia pages. Thus, there is right-tail censoring in this variable,
and this should be better understood in future.

Additionally, nearly half of the albums pulled have 0 review scores.
Given 12,694 albums and 1,000 artists, we have on average about 12
albums per artist. This is somewhat high, and suggests there may be the
presence of live albums, EPs, etc that are lesser known and less
reviewed. I also note the presence of classical musicians and music
associated with children's cartoons, that are popular but not in scope
of the study. For now, I drop all 0's, under the assumption that popular
artists in scope for analysis would likely have all of their major
albums reviewed.

\begin{Shaded}
\begin{Highlighting}[]
\NormalTok{cleandata }\OtherTok{=}\NormalTok{ finaldata }\SpecialCharTok{\%\textgreater{}\%}
  \FunctionTok{filter}\NormalTok{(numscores}\SpecialCharTok{\textgreater{}}\DecValTok{0}\NormalTok{)}
\end{Highlighting}
\end{Shaded}

Next we look at the number of albums by artist:

\begin{figure}
\centering
\includegraphics{__PAPER___files/figure-latex/unnamed-chunk-20-1.pdf}
\caption{Distribution of Number of Albums}
\end{figure}

We note first that there are several extreme outliers who have up to
nearly 250 albums. Let's take a look at a few of the most extreme
outliers:

\begin{table}[H]

\caption{\label{tab:unnamed-chunk-21}Artists with More than 50 Albums}
\centering
\begin{tabular}[t]{l|r}
\hline
Artist Name & Num. Albums\\
\hline
Elvis Presley & 54\\
\hline
Frank Sinatra & 64\\
\hline
Johnny Cash & 78\\
\hline
Bob Dylan & 57\\
\hline
Frédéric Chopin & 245\\
\hline
Neil Young & 67\\
\hline
Dolly Parton & 65\\
\hline
Willie Nelson & 85\\
\hline
Ella Fitzgerald & 78\\
\hline
Grateful Dead & 64\\
\hline
\end{tabular}
\end{table}

Frederic Chopin, a classical composer, is a clear outlier and not in
scope. I therefore drop from subsequent analysis. Other outliers do
appear however to be in scope. The next biggest outlier is Willie Nelson
with 85 albums. Let's take a look at the distribution of the number of
scrobbles of Willie's albums:

\begin{figure}
\centering
\includegraphics{__PAPER___files/figure-latex/unnamed-chunk-23-1.pdf}
\caption{Distribution of Scrobbles Among Willie Nelson Albums}
\end{figure}

Willie Nelson appears to have quite a few albums with very few
scrobbles. These likely represent more niche releases such as live
albums or demos. While not necessarily out of scope, it would be worth
doing a robustness check in the analysis stage where these are dropped.
For now I use a simple method: classify the smallest 2.5\% of an
artist's albums as ``outliers''.

\begin{Shaded}
\begin{Highlighting}[]
\NormalTok{cleandata }\OtherTok{=}\NormalTok{ cleandata }\SpecialCharTok{\%\textgreater{}\%}
  \FunctionTok{group\_by}\NormalTok{(artist\_id) }\SpecialCharTok{\%\textgreater{}\%}
  \FunctionTok{mutate}\NormalTok{(}\AttributeTok{lowerbound=}\FunctionTok{quantile}\NormalTok{(album\_scrobbles,}\AttributeTok{p=}\FloatTok{0.025}\NormalTok{,}\AttributeTok{na.rm=}\NormalTok{T)) }\SpecialCharTok{\%\textgreater{}\%}
  \FunctionTok{ungroup}\NormalTok{() }\SpecialCharTok{\%\textgreater{}\%}
  \FunctionTok{mutate}\NormalTok{(}\AttributeTok{outlier=}\NormalTok{album\_scrobbles}\SpecialCharTok{\textless{}}\NormalTok{lowerbound) }
\end{Highlighting}
\end{Shaded}

We can see that in the case of Willie Nelson, we would drop three albums
with only a handful of scrobbles.

\begin{Shaded}
\begin{Highlighting}[]
\NormalTok{cleandata }\SpecialCharTok{\%\textgreater{}\%}
  \FunctionTok{filter}\NormalTok{(artist\_name}\SpecialCharTok{==}\StringTok{"Willie Nelson"}\NormalTok{) }\SpecialCharTok{\%\textgreater{}\%}
\NormalTok{  dplyr}\SpecialCharTok{::}\FunctionTok{select}\NormalTok{(}\FunctionTok{c}\NormalTok{(artist\_name,album\_name,album\_scrobbles,outlier)) }\SpecialCharTok{\%\textgreater{}\%}
  \FunctionTok{filter}\NormalTok{(outlier}\SpecialCharTok{==}\ConstantTok{TRUE}\NormalTok{) }\SpecialCharTok{\%\textgreater{}\%}
  \FunctionTok{kable}\NormalTok{(}\AttributeTok{caption=}\StringTok{"Outlier Example: Willie Nelson"}
\NormalTok{        ,}\AttributeTok{col.names=}\FunctionTok{c}\NormalTok{(}\StringTok{"Artist"}\NormalTok{,}\StringTok{"Album"}\NormalTok{,}\StringTok{"Scrobbles"}\NormalTok{,}\StringTok{"Outlier?"}\NormalTok{)) }\SpecialCharTok{\%\textgreater{}\%}
  \FunctionTok{kable\_styling}\NormalTok{(}\AttributeTok{latex\_options =} \StringTok{"HOLD\_position"}\NormalTok{)}
\end{Highlighting}
\end{Shaded}

\begin{table}[H]

\caption{\label{tab:unnamed-chunk-25}Outlier Example: Willie Nelson}
\centering
\begin{tabular}[t]{l|l|r|l}
\hline
Artist & Album & Scrobbles & Outlier?\\
\hline
Willie Nelson & If I Can Find a Clean Shirt & 93 & TRUE\\
\hline
Willie Nelson & The Winning Hand & 270 & TRUE\\
\hline
Willie Nelson & WWII & 149 & TRUE\\
\hline
\end{tabular}
\end{table}

Finally, I drop a few albums with missing or 0 scrobbles.

\begin{Shaded}
\begin{Highlighting}[]
\NormalTok{cleandata }\OtherTok{=}\NormalTok{ cleandata }\SpecialCharTok{\%\textgreater{}\%}
  \FunctionTok{filter}\NormalTok{(}\SpecialCharTok{!}\FunctionTok{is.na}\NormalTok{(album\_scrobbles)) }\SpecialCharTok{\%\textgreater{}\%}
  \FunctionTok{filter}\NormalTok{(album\_scrobbles}\SpecialCharTok{\textgreater{}}\DecValTok{0}\NormalTok{)}
\end{Highlighting}
\end{Shaded}

The resulting dataset has 755 artists and 6193 albums. Below are summary
tables of artist-level and album-level statistics.

\begin{table}[H]

\caption{\label{tab:unnamed-chunk-27}Artist-level Summary Statistics}
\centering
\begin{tabular}[t]{llllllll}
\toprule
Variable & N & Mean & Std. Dev. & Min & Pctl. 25 & Pctl. 75 & Max\\
\midrule
Artist Scrobbles & 755 & 25642436.976 & 46461607.397 & 13 & 2743014.5 & 27194475.5 & 556318942\\
Num. Albums & 755 & 8.203 & 9.879 & 1 & 2 & 10 & 85\\
Num. Followers & 755 & 6158321.025 & 8536062.242 & 3790 & 2044827.5 & 6485957.5 & 88706690\\
\bottomrule
\end{tabular}
\end{table}

\begin{table}[H]

\caption{\label{tab:unnamed-chunk-28}Album-level Summary Statistics}
\centering
\begin{tabular}[t]{llllllll}
\toprule
Variable & N & Mean & Std. Dev. & Min & Pctl. 25 & Pctl. 75 & Max\\
\midrule
Album Scrobbles &  &  &  &  &  &  & \\
Num. Review Scores & 6193 & 5.347 & 3.566 & 1 & 2 & 9 & 27\\
Avg. Review Scores & 6118 & 69.705 & 13.067 & 13.333 & 60 & 79.5 & 100\\
\bottomrule
\end{tabular}
\end{table}

\hypertarget{model}{%
\section{Model}\label{model}}

\hypertarget{model-1-linear-model-with-fixed-effects}{%
\subsection{Model 1: Linear model with fixed
effects}\label{model-1-linear-model-with-fixed-effects}}

Let \(s_{ijt}\) denote the number of scrobbles that artist \(i\) has
gotten on album \(j\) after \(t\) days have occured--i.e.~scrobbles are
cumulative over time. Then we model scrobbles as follows:

\[s_{ijt} = \alpha_i + \beta_t t + \beta_r r_{ij} + \epsilon_{ijt}\]
where \(\alpha_0, \alpha_i, \beta_t, \beta_r\) are parameters,
\(\alpha_i\) is an artist-specific fixed effect, \(r_{ij}\) is an
aggregated measure of review scores, and \(\epsilon_{ijt}\) is a random
innovation. The main advantage of this model is the inclusion of artist
fixed effects, which account for unobserved artist-specific factors--for
example, and artists inherent popularity.

\hypertarget{model-2-log-transformed}{%
\subsection{Model 2: Log-transformed}\label{model-2-log-transformed}}

It is reasonable to think that the effect of review scores on scrobbles
could be based on percentage changes rather than linear changes. In such
a case, log transforming the key variables of interest may produce more
realistic results:

\[\log s_{ijt} = \alpha_i + \beta_t \log t + \beta_r \log r_{ij} + \epsilon_{ijt}\]

\hypertarget{model-3-legacy-artists}{%
\subsection{Model 3: Legacy artists}\label{model-3-legacy-artists}}

As discussed in the introduction, one method I can use to account for
endogeneity is to assume that correlations between album performance and
review scores for legacy albums--that is, albums released prior to the
existence of streaming platforms--is spurious. In other words, reviews
of the album predict audience reception rather than influence audience
behaviour. In this way, differences between legacy albums and newer
albums should reflect the causal portion of \(\beta_r\), the parameter
of interest.

This is a diff-in-diff approach, and is modelled as follows. Let
\(L_i=1\) if artist \(i\) is a legacy artist and 0 otherwise. Then,

\[\log s_{ijt} = \alpha_i + \beta_t \log t + \beta_0 L_i + \beta_r (1-L_i) \log r_{ij} + \epsilon_{ijt}\]
As for how to define legacy albums, for now I use the cutoff of albums
released prior to 2010, as Spotify grew its user base in the early
2010's.

\hypertarget{note-on-standard-errors}{%
\subsection{Note on Standard Errors}\label{note-on-standard-errors}}

Note that it is likely that our data is clustered--that is, scrobbles
from different albums of the same artist are correlated. While OLS
estimates are still unbiased under this assumption, standard errors will
be too small. Thus, I use cluster-robust standard errors in all
analysis.

\hypertarget{results}{%
\section{Results}\label{results}}

\hypertarget{estimation}{%
\subsection{Estimation}\label{estimation}}

Before estimation, I define a few variables:

\begin{Shaded}
\begin{Highlighting}[]
\CommentTok{\# num days since album release}
\NormalTok{estdata }\OtherTok{=}\NormalTok{ cleandata }\SpecialCharTok{\%\textgreater{}\%}
  \FunctionTok{mutate}\NormalTok{(}\AttributeTok{releasedate =} \FunctionTok{as.Date}\NormalTok{(album\_releasedate)) }\SpecialCharTok{\%\textgreater{}\%}
  \FunctionTok{mutate}\NormalTok{(}\AttributeTok{currdate =} \FunctionTok{ifelse}\NormalTok{(batch}\SpecialCharTok{==}\DecValTok{1}\NormalTok{,}\StringTok{"2021{-}12{-}05"}\NormalTok{,}\StringTok{"2021{-}12{-}15"}\NormalTok{)) }\SpecialCharTok{\%\textgreater{}\%}
  \FunctionTok{mutate}\NormalTok{(}\AttributeTok{currdate =} \FunctionTok{as.Date}\NormalTok{(currdate)) }\SpecialCharTok{\%\textgreater{}\%}
  \FunctionTok{mutate}\NormalTok{(}\AttributeTok{numdays =} \FunctionTok{as.numeric}\NormalTok{(currdate }\SpecialCharTok{{-}}\NormalTok{ releasedate))}

\CommentTok{\# classify legacy album}
\NormalTok{estdata }\OtherTok{=}\NormalTok{ estdata }\SpecialCharTok{\%\textgreater{}\%}
  \FunctionTok{mutate}\NormalTok{(}\AttributeTok{legacy =} \FunctionTok{as.numeric}\NormalTok{(releasedate}\SpecialCharTok{\textless{}}\FunctionTok{as.Date}\NormalTok{(}\StringTok{"2010{-}01{-}01"}\NormalTok{)))}
\end{Highlighting}
\end{Shaded}

Here I estimate each of the three models:

\begin{Shaded}
\begin{Highlighting}[]
\NormalTok{model1 }\OtherTok{=} \FunctionTok{plm}\NormalTok{(album\_scrobbles }\SpecialCharTok{\textasciitilde{}}\NormalTok{ numdays }\SpecialCharTok{+}\NormalTok{ meanscores}
\NormalTok{          ,}\AttributeTok{data=}\NormalTok{estdata}
\NormalTok{          ,}\AttributeTok{index=}\StringTok{"artist\_id"}
\NormalTok{          ,}\AttributeTok{model=}\StringTok{"within"}\NormalTok{)}
\NormalTok{cov1       }\OtherTok{=} \FunctionTok{vcovHC}\NormalTok{(model1,}\AttributeTok{method=}\StringTok{"arellano"}\NormalTok{,}\AttributeTok{type=}\StringTok{"HC1"}\NormalTok{,}\AttributeTok{cluster=}\StringTok{"group"}\NormalTok{)}
\NormalTok{robust.se1 }\OtherTok{=} \FunctionTok{sqrt}\NormalTok{(}\FunctionTok{diag}\NormalTok{(cov1))}

\NormalTok{model2 }\OtherTok{=} \FunctionTok{plm}\NormalTok{(}\FunctionTok{log}\NormalTok{(album\_scrobbles) }\SpecialCharTok{\textasciitilde{}} \FunctionTok{log}\NormalTok{(numdays) }\SpecialCharTok{+} \FunctionTok{log}\NormalTok{(meanscores)}
\NormalTok{          ,}\AttributeTok{data=}\NormalTok{estdata}
\NormalTok{          ,}\AttributeTok{index=}\StringTok{"artist\_id"}
\NormalTok{          ,}\AttributeTok{model=}\StringTok{"within"}\NormalTok{)}
\NormalTok{cov2       }\OtherTok{=} \FunctionTok{vcovHC}\NormalTok{(model2,}\AttributeTok{method=}\StringTok{"arellano"}\NormalTok{,}\AttributeTok{type=}\StringTok{"HC1"}\NormalTok{,}\AttributeTok{cluster=}\StringTok{"group"}\NormalTok{)}
\NormalTok{robust.se2 }\OtherTok{=} \FunctionTok{sqrt}\NormalTok{(}\FunctionTok{diag}\NormalTok{(cov2))}

\NormalTok{estdata }\OtherTok{=}\NormalTok{ estdata }\SpecialCharTok{\%\textgreater{}\%} \FunctionTok{mutate}\NormalTok{(}\AttributeTok{logscores\_diffindiff =}\NormalTok{ (}\DecValTok{1}\SpecialCharTok{{-}}\NormalTok{legacy)}\SpecialCharTok{*}\FunctionTok{log}\NormalTok{(meanscores))}
\NormalTok{model3 }\OtherTok{=} \FunctionTok{plm}\NormalTok{(}\FunctionTok{log}\NormalTok{(album\_scrobbles) }\SpecialCharTok{\textasciitilde{}} \FunctionTok{log}\NormalTok{(numdays) }\SpecialCharTok{+}\NormalTok{ legacy }\SpecialCharTok{+}\NormalTok{ logscores\_diffindiff}
\NormalTok{          ,}\AttributeTok{data=}\NormalTok{estdata}
\NormalTok{          ,}\AttributeTok{index=}\StringTok{"artist\_id"}
\NormalTok{          ,}\AttributeTok{model=}\StringTok{"within"}\NormalTok{)}
\NormalTok{cov3       }\OtherTok{=} \FunctionTok{vcovHC}\NormalTok{(model3,}\AttributeTok{method=}\StringTok{"arellano"}\NormalTok{,}\AttributeTok{type=}\StringTok{"HC1"}\NormalTok{,}\AttributeTok{cluster=}\StringTok{"group"}\NormalTok{)}
\NormalTok{robust.se3 }\OtherTok{=} \FunctionTok{sqrt}\NormalTok{(}\FunctionTok{diag}\NormalTok{(cov3))}
\end{Highlighting}
\end{Shaded}

\begin{Shaded}
\begin{Highlighting}[]
\FunctionTok{stargazer}\NormalTok{(model1,model2,model3}
\NormalTok{          ,}\AttributeTok{se=}\FunctionTok{list}\NormalTok{(robust.se1,robust.se2,robust.se3)}
\NormalTok{          ,}\AttributeTok{title=}\StringTok{"Regression Results"}\NormalTok{)}
\end{Highlighting}
\end{Shaded}

\% Table created by stargazer v.5.2.2 by Marek Hlavac, Harvard
University. E-mail: hlavac at fas.harvard.edu \% Date and time: Thu, Dec
16, 2021 - 11:41:09 PM

\begin{table}[!htbp] \centering 
  \caption{Regression Results} 
  \label{} 
\begin{tabular}{@{\extracolsep{5pt}}lccc} 
\\[-1.8ex]\hline 
\hline \\[-1.8ex] 
 & \multicolumn{3}{c}{\textit{Dependent variable:}} \\ 
\cline{2-4} 
\\[-1.8ex] & album\_scrobbles & \multicolumn{2}{c}{log(album\_scrobbles)} \\ 
\\[-1.8ex] & (1) & (2) & (3)\\ 
\hline \\[-1.8ex] 
 numdays & 142.101$^{***}$ &  &  \\ 
  & (21.233) &  &  \\ 
  & & & \\ 
 meanscores & 59,910.570$^{***}$ &  &  \\ 
  & (9,031.494) &  &  \\ 
  & & & \\ 
 log(numdays) &  & 0.493$^{***}$ & 0.447$^{***}$ \\ 
  &  & (0.031) & (0.035) \\ 
  & & & \\ 
 log(meanscores) &  & 1.939$^{***}$ &  \\ 
  &  & (0.150) &  \\ 
  & & & \\ 
 legacy &  &  & 7.417$^{***}$ \\ 
  &  &  & (1.346) \\ 
  & & & \\ 
 logscores\_diffindiff &  &  & 1.711$^{***}$ \\ 
  &  &  & (0.319) \\ 
  & & & \\ 
\hline \\[-1.8ex] 
Observations & 5,688 & 5,688 & 5,688 \\ 
R$^{2}$ & 0.021 & 0.105 & 0.078 \\ 
Adjusted R$^{2}$ & $-$0.127 & $-$0.031 & $-$0.062 \\ 
F Statistic & 54.181$^{***}$ (df = 2; 4938) & 289.532$^{***}$ (df = 2; 4938) & 139.580$^{***}$ (df = 3; 4937) \\ 
\hline 
\hline \\[-1.8ex] 
\textit{Note:}  & \multicolumn{3}{r}{$^{*}$p$<$0.1; $^{**}$p$<$0.05; $^{***}$p$<$0.01} \\ 
\end{tabular} 
\end{table}

\hypertarget{interpretation-of-results}{%
\subsection{Interpretation of Results}\label{interpretation-of-results}}

Table 5 presents results from each of the three models. I note first
that in all models, average review scores are statistically significant
predictors of number of scrobbles at 1\%. To interpet the model, let's
use an example of an album that has 100,000 scrobbles and an average
review score of 50. If they increased their review score to an 80, model
1 predicts they would gain about 1.8 million scrobbles. Model 2 predicts
they would gain about 250,000, and Model 3 predicts they would gain
about 220,000.

Thus, we see first that the log-transformed version provides more
realistic predictions as expected. This is also reflected in better
model fit statistcs. Furthermore, we observe that the diff-in-diff
version predicts a smaller gain, though not much different from the
model without this component.

\hypertarget{conclusions-future-work}{%
\section{Conclusions / Future Work}\label{conclusions-future-work}}

At the onset of this project, I started with three primary research
questions:

\begin{itemize}
\tightlist
\item
  Do music critics affect album performance?
\item
  Differences between user and expert scores?
\item
  Difference in new and old media?
\end{itemize}

This paper does some investigation into the first question, although
time and data constraints preclude me from looking at the second and
third. After accounting for artist fixed effects, and using a
diff-in-diff approach to account for spurious correlations, we find that
review scores do seem to have an effect on album performance, at least
based on our current data and assumptions.

Future work should focus on performing robustness checks on my current
analysis. Furthermore, additional controls could be included in the
model. Easy choices using similar data approaches would be track
information (e.g.~perhaps ``happier'' albums get more listens AND better
reviews) from the Spotify API and demographic information about the
artists from their Wikipedia pages.

Additional work could be directed at fine tuning my data algorithm.
While I know by example that it seems to work in many cases, it would be
great to do a systematic analysis to determine how effective my method
is. For example, what is the nature of the nearly half of albums that
had no reviews?

Further, I would reconsider how to overcome data limitations with
respect to listening numbers (Last.FM has potential selection issues as
discussed) and user scores (to analyze user vs expert score effects).
The former could be purchased from chartmasters.org, while the latter
theoretically exists on metacritic--perhaps one day they will allow
access to the data via API.

Finally, in the long term it would be interesting to merge this data
with review scores from theneedledrop's YouTube channel to study old vs
new media effects, particularly by genre. theneedledrop puts his review
scores in the description of his videos, allowing one in theory to
access them via YouTube's API.

\newpage

\hypertarget{appendix-i-web-scraping-critic-and-user-scores}{%
\section{Appendix I: Web Scraping Critic and User
Scores}\label{appendix-i-web-scraping-critic-and-user-scores}}

Here is code to scrape critic and user scores from Metacritic, using
Drake's Certified Lover Boy as an example. I include it here to document
my work, but do NOT suggest using it on a mass scale as web scraping is
prohibited by their terms of service.

\begin{Shaded}
\begin{Highlighting}[]
\CommentTok{\# get album name in right format}
\NormalTok{album\_name\_meta }\OtherTok{=} \FunctionTok{tolower}\NormalTok{(tm}\SpecialCharTok{::}\FunctionTok{removePunctuation}\NormalTok{(album\_name))}
\NormalTok{album\_name\_meta }\OtherTok{=}\NormalTok{ album\_name\_meta }\SpecialCharTok{\%\textgreater{}\%}
  \FunctionTok{str\_replace\_all}\NormalTok{(}\StringTok{"\&"}\NormalTok{, }\StringTok{""}\NormalTok{) }\SpecialCharTok{\%\textgreater{}\%}
  \FunctionTok{tolower}\NormalTok{() }\SpecialCharTok{\%\textgreater{}\%}
  \FunctionTok{str\_squish}\NormalTok{() }\SpecialCharTok{\%\textgreater{}\%}
  \FunctionTok{str\_replace\_all}\NormalTok{(}\StringTok{" "}\NormalTok{, }\StringTok{"{-}"}\NormalTok{)}

\CommentTok{\# get artists name in right format}
\NormalTok{arist\_name\_meta }\OtherTok{=} \FunctionTok{tolower}\NormalTok{(}\FunctionTok{removePunctuation}\NormalTok{(artist\_name))}
\NormalTok{artist\_name\_meta }\OtherTok{=}\NormalTok{ arist\_name\_meta }\SpecialCharTok{\%\textgreater{}\%}
  \FunctionTok{str\_replace\_all}\NormalTok{(}\StringTok{"\&"}\NormalTok{, }\StringTok{""}\NormalTok{) }\SpecialCharTok{\%\textgreater{}\%}
  \FunctionTok{tolower}\NormalTok{() }\SpecialCharTok{\%\textgreater{}\%}
  \FunctionTok{str\_squish}\NormalTok{() }\SpecialCharTok{\%\textgreater{}\%}
  \FunctionTok{str\_replace\_all}\NormalTok{(}\StringTok{" "}\NormalTok{, }\StringTok{"{-}"}\NormalTok{)}

\CommentTok{\# critic scores}
\CommentTok{\# read in html}
\NormalTok{url\_critic }\OtherTok{=} \FunctionTok{paste0}\NormalTok{(}\StringTok{\textquotesingle{}https://www.metacritic.com/music/\textquotesingle{}}
\NormalTok{              ,album\_name\_meta,}\StringTok{\textquotesingle{}/\textquotesingle{}}
\NormalTok{              ,artist\_name\_meta}
\NormalTok{              ,}\StringTok{\textquotesingle{}/critic{-}reviews\textquotesingle{}}\NormalTok{)}
\NormalTok{url\_user   }\OtherTok{=} \FunctionTok{paste0}\NormalTok{(}\StringTok{\textquotesingle{}https://www.metacritic.com/music/\textquotesingle{}}
\NormalTok{              ,album\_name\_meta,}\StringTok{\textquotesingle{}/\textquotesingle{}}
\NormalTok{              ,artist\_name\_meta}
\NormalTok{              ,}\StringTok{\textquotesingle{}/user{-}reviews\textquotesingle{}}\NormalTok{)}

\ControlFlowTok{if}\NormalTok{ (}\SpecialCharTok{!}\FunctionTok{http\_error}\NormalTok{(url\_critic)) \{}
  
\NormalTok{  webpage }\OtherTok{=} \FunctionTok{read\_html}\NormalTok{(url\_critic)}

  \CommentTok{\# critic scores}
\NormalTok{  xpath }\OtherTok{=} \StringTok{\textquotesingle{}//*[(@id = "main")]//*[contains(concat( " ", @class, " " ), concat( " ", "indiv", " " ))]\textquotesingle{}}
\NormalTok{  criticscores }\OtherTok{=} \FunctionTok{html\_nodes}\NormalTok{(webpage, }\AttributeTok{xpath=}\NormalTok{xpath)}
\NormalTok{  criticscores }\OtherTok{=} \FunctionTok{html\_text}\NormalTok{(criticscores,}\AttributeTok{trim=}\NormalTok{T)}
\NormalTok{  criticscores }\OtherTok{=} \FunctionTok{as.numeric}\NormalTok{(criticscores)}

  \CommentTok{\# user scores}
  \CommentTok{\# read the aggregated score}
\NormalTok{  webpage }\OtherTok{=} \FunctionTok{read\_html}\NormalTok{(url\_user)}
\NormalTok{  xpath }\OtherTok{=} \StringTok{\textquotesingle{}//*[contains(concat( " ", @class, " " ), concat( " ", "large", " " ))]\textquotesingle{}}
\NormalTok{  userscore\_agg }\OtherTok{=} \FunctionTok{html\_nodes}\NormalTok{(webpage, }\AttributeTok{xpath=}\NormalTok{xpath)}
\NormalTok{  userscore\_agg }\OtherTok{=} \FunctionTok{html\_text}\NormalTok{(userscore\_agg,}\AttributeTok{trim=}\NormalTok{T)}
\NormalTok{  userscore\_agg }\OtherTok{=} \FunctionTok{as.numeric}\NormalTok{(userscore\_agg[}\DecValTok{1}\NormalTok{])}

  \CommentTok{\# read number positives, neutral, negative}
\NormalTok{  xpath}\OtherTok{=}\StringTok{\textquotesingle{}//*[(@id = "main")]//*[contains(concat( " ", @class, " " ), concat( " ", "count", " " ))]\textquotesingle{}}
\NormalTok{  userscore\_meta }\OtherTok{=} \FunctionTok{html\_nodes}\NormalTok{(webpage, }\AttributeTok{xpath=}\NormalTok{xpath)}
\NormalTok{  userscore\_meta }\OtherTok{=} \FunctionTok{html\_text}\NormalTok{(userscore\_meta,}\AttributeTok{trim=}\NormalTok{T)}
\NormalTok{  userscore\_meta }\OtherTok{=} \FunctionTok{as.numeric}\NormalTok{(userscore\_meta[}\DecValTok{2}\SpecialCharTok{:}\DecValTok{4}\NormalTok{])}
\NormalTok{\}}

\NormalTok{criticscores}
\end{Highlighting}
\end{Shaded}

\begin{verbatim}
##  [1] 80 72 70 68 67 67 66 60 60 60 60 60 50 50 45 40 40 40 40 40
\end{verbatim}

\begin{Shaded}
\begin{Highlighting}[]
\NormalTok{userscore\_agg}
\end{Highlighting}
\end{Shaded}

\begin{verbatim}
## [1] 3.5
\end{verbatim}

\begin{Shaded}
\begin{Highlighting}[]
\NormalTok{userscore\_meta}
\end{Highlighting}
\end{Shaded}

\begin{verbatim}
## [1] 237 186 655
\end{verbatim}

\newpage

\hypertarget{appendix-ii-scraping-wikipedia-articles-for-critic-scores}{%
\section{Appendix II: Scraping Wikipedia Articles for Critic
Scores}\label{appendix-ii-scraping-wikipedia-articles-for-critic-scores}}

\begin{Shaded}
\begin{Highlighting}[]
\CommentTok{\# Function for converting letter grades to number grades}
\NormalTok{lettergrade\_num }\OtherTok{=} \ControlFlowTok{function}\NormalTok{(x) \{}
  \ControlFlowTok{if}\NormalTok{ (x}\SpecialCharTok{==}\StringTok{"A+"} \SpecialCharTok{|}\NormalTok{ x}\SpecialCharTok{==}\StringTok{"A"}\NormalTok{) \{}
    \FunctionTok{return}\NormalTok{(}\StringTok{"100"}\NormalTok{)}
\NormalTok{  \} }\ControlFlowTok{else} \ControlFlowTok{if}\NormalTok{ (x}\SpecialCharTok{==}\StringTok{"A{-}"}\NormalTok{) \{}
    \FunctionTok{return}\NormalTok{(}\StringTok{"91"}\NormalTok{)}
\NormalTok{  \} }\ControlFlowTok{else} \ControlFlowTok{if}\NormalTok{ (x}\SpecialCharTok{==}\StringTok{"B+"}\NormalTok{) \{}
    \FunctionTok{return}\NormalTok{(}\StringTok{"83"}\NormalTok{)}
\NormalTok{  \} }\ControlFlowTok{else} \ControlFlowTok{if}\NormalTok{ (x}\SpecialCharTok{==}\StringTok{"B"}\NormalTok{) \{}
    \FunctionTok{return}\NormalTok{(}\StringTok{"75"}\NormalTok{)}
\NormalTok{  \} }\ControlFlowTok{else} \ControlFlowTok{if}\NormalTok{ (x}\SpecialCharTok{==}\StringTok{"B{-}"}\NormalTok{) \{}
    \FunctionTok{return}\NormalTok{(}\StringTok{"67"}\NormalTok{)}
\NormalTok{  \} }\ControlFlowTok{else} \ControlFlowTok{if}\NormalTok{ (x}\SpecialCharTok{==}\StringTok{"C+"}\NormalTok{) \{}
    \FunctionTok{return}\NormalTok{(}\StringTok{"58"}\NormalTok{)}
\NormalTok{  \} }\ControlFlowTok{else} \ControlFlowTok{if}\NormalTok{ (x}\SpecialCharTok{==}\StringTok{"C"}\NormalTok{) \{}
    \FunctionTok{return}\NormalTok{(}\StringTok{"50"}\NormalTok{)}
\NormalTok{  \} }\ControlFlowTok{else} \ControlFlowTok{if}\NormalTok{ (x}\SpecialCharTok{==}\StringTok{"C{-}"}\NormalTok{) \{}
    \FunctionTok{return}\NormalTok{(}\StringTok{"42"}\NormalTok{)}
\NormalTok{  \} }\ControlFlowTok{else} \ControlFlowTok{if}\NormalTok{ (x}\SpecialCharTok{==}\StringTok{"D+"}\NormalTok{) \{}
    \FunctionTok{return}\NormalTok{(}\StringTok{"33"}\NormalTok{)}
\NormalTok{  \} }\ControlFlowTok{else} \ControlFlowTok{if}\NormalTok{ (x}\SpecialCharTok{==}\StringTok{"D"}\NormalTok{) \{}
    \FunctionTok{return}\NormalTok{(}\StringTok{"25"}\NormalTok{)}
\NormalTok{  \} }\ControlFlowTok{else} \ControlFlowTok{if}\NormalTok{ (x}\SpecialCharTok{==}\StringTok{"D{-}"}\NormalTok{) \{}
    \FunctionTok{return}\NormalTok{(}\StringTok{"16"}\NormalTok{)}
\NormalTok{  \} }\ControlFlowTok{else} \ControlFlowTok{if}\NormalTok{ (x}\SpecialCharTok{==}\StringTok{"F+"}\NormalTok{) \{}
    \FunctionTok{return}\NormalTok{(}\StringTok{"8"}\NormalTok{)}
\NormalTok{  \} }\ControlFlowTok{else} \ControlFlowTok{if}\NormalTok{ (x}\SpecialCharTok{==}\StringTok{"F"} \SpecialCharTok{|}\NormalTok{ x}\SpecialCharTok{==}\StringTok{"F{-}"}\NormalTok{) \{}
    \FunctionTok{return}\NormalTok{(}\StringTok{"0"}\NormalTok{)}
\NormalTok{  \} }\ControlFlowTok{else}\NormalTok{ \{}
    \FunctionTok{return}\NormalTok{(x)}
\NormalTok{  \}}
\NormalTok{\}}

\CommentTok{\# get web page content}
\NormalTok{webpage }\OtherTok{=} \FunctionTok{read\_html}\NormalTok{(webpage}\SpecialCharTok{$}\NormalTok{parse}\SpecialCharTok{$}\NormalTok{text}\SpecialCharTok{$}\StringTok{\textasciigrave{}}\AttributeTok{*}\StringTok{\textasciigrave{}}\NormalTok{)}
\NormalTok{xpath}\OtherTok{=}\StringTok{\textquotesingle{}//*[contains(concat( " ", @class, " " ), concat( " ", "floatright", " " ))]//td\textquotesingle{}}
\NormalTok{wikiscores }\OtherTok{=} \FunctionTok{html\_nodes}\NormalTok{(webpage, }\AttributeTok{xpath=}\NormalTok{xpath)}
\NormalTok{wikiscores\_stars }\OtherTok{=} \FunctionTok{html\_attr}\NormalTok{(}\FunctionTok{html\_element}\NormalTok{(wikiscores,}\AttributeTok{css=}\StringTok{\textquotesingle{}span\textquotesingle{}}\NormalTok{),}\StringTok{\textquotesingle{}title\textquotesingle{}}\NormalTok{)}
\NormalTok{wikiscores }\OtherTok{=} \FunctionTok{html\_text}\NormalTok{(wikiscores,}\AttributeTok{trim=}\NormalTok{T)}

  \CommentTok{\# Convert scores to numeric out of 100}
\NormalTok{  wikiscores\_data }\OtherTok{=} \FunctionTok{as.data.frame}\NormalTok{(}\FunctionTok{matrix}\NormalTok{(wikiscores,}\AttributeTok{ncol=}\DecValTok{2}\NormalTok{,}\AttributeTok{byrow=}\NormalTok{T))}
\NormalTok{  wikiscores\_stars\_data }\OtherTok{=} \FunctionTok{as.data.frame}\NormalTok{(}\FunctionTok{matrix}\NormalTok{(wikiscores\_stars,}\AttributeTok{ncol=}\DecValTok{2}\NormalTok{,}\AttributeTok{byrow=}\NormalTok{T))}
  \CommentTok{\# deal with specific review websites}
\NormalTok{  selector }\OtherTok{=}\NormalTok{ (}\SpecialCharTok{!}\FunctionTok{is.na}\NormalTok{(wikiscores\_stars\_data}\SpecialCharTok{$}\NormalTok{V2) }\SpecialCharTok{\&} \SpecialCharTok{!}\NormalTok{(}\FunctionTok{tolower}\NormalTok{(wikiscores\_data}\SpecialCharTok{$}\NormalTok{V1)}\SpecialCharTok{==}\StringTok{"tom hull – on the Web"}\NormalTok{))}
\NormalTok{  wikiscores\_data}\SpecialCharTok{$}\NormalTok{V2[selector] }\OtherTok{=}\NormalTok{ wikiscores\_stars\_data}\SpecialCharTok{$}\NormalTok{V2[selector]}
\NormalTok{  wikiscores\_data }\OtherTok{=}\NormalTok{ wikiscores\_data[wikiscores\_data}\SpecialCharTok{$}\NormalTok{V1}\SpecialCharTok{!=}\StringTok{"Metacritic"}\NormalTok{,]}
\NormalTok{  wikiscores\_data }\OtherTok{=}\NormalTok{ wikiscores\_data[wikiscores\_data}\SpecialCharTok{$}\NormalTok{V1}\SpecialCharTok{!=}\StringTok{"AnyDecentMusic?"}\NormalTok{,]}
  \CommentTok{\# remove [] and content inside}
\NormalTok{  wikiscores\_data}\SpecialCharTok{$}\NormalTok{V2 }\OtherTok{=} \FunctionTok{gsub}\NormalTok{(}\StringTok{"}\SpecialCharTok{\textbackslash{}\textbackslash{}}\StringTok{[.*?}\SpecialCharTok{\textbackslash{}\textbackslash{}}\StringTok{]"}\NormalTok{,}\StringTok{""}\NormalTok{,wikiscores\_data}\SpecialCharTok{$}\NormalTok{V2)}
  \CommentTok{\# remove () but NOT CONTENT (sometimes the score is contained within () )}
\NormalTok{  wikiscores\_data}\SpecialCharTok{$}\NormalTok{V2 }\OtherTok{=} \FunctionTok{str\_replace}\NormalTok{(wikiscores\_data}\SpecialCharTok{$}\NormalTok{V2,}\StringTok{"}\SpecialCharTok{\textbackslash{}\textbackslash{}}\StringTok{("}\NormalTok{,}\StringTok{""}\NormalTok{)}
\NormalTok{  wikiscores\_data}\SpecialCharTok{$}\NormalTok{V2 }\OtherTok{=} \FunctionTok{str\_replace}\NormalTok{(wikiscores\_data}\SpecialCharTok{$}\NormalTok{V2,}\StringTok{"}\SpecialCharTok{\textbackslash{}\textbackslash{}}\StringTok{)"}\NormalTok{,}\StringTok{""}\NormalTok{)}
  \CommentTok{\# fix ill{-}behaved special characters}
\NormalTok{  wikiscores\_data}\SpecialCharTok{$}\NormalTok{V2 }\OtherTok{=} \FunctionTok{sapply}\NormalTok{(wikiscores\_data}\SpecialCharTok{$}\NormalTok{V2 , }\ControlFlowTok{function}\NormalTok{(x) }\FunctionTok{URLencode}\NormalTok{(x) )}
\NormalTok{  wikiscores\_data}\SpecialCharTok{$}\NormalTok{V2 }\OtherTok{=} \FunctionTok{str\_replace}\NormalTok{(wikiscores\_data}\SpecialCharTok{$}\NormalTok{V2,}\StringTok{"\%E2\%88\%92"}\NormalTok{,}\StringTok{"{-}"}\NormalTok{)}
\NormalTok{  wikiscores\_data}\SpecialCharTok{$}\NormalTok{V2 }\OtherTok{=} \FunctionTok{str\_replace}\NormalTok{(wikiscores\_data}\SpecialCharTok{$}\NormalTok{V2,}\StringTok{"\%E2\%80\%93"}\NormalTok{,}\StringTok{"{-}"}\NormalTok{)}
\NormalTok{  wikiscores\_data}\SpecialCharTok{$}\NormalTok{V2 }\OtherTok{=} \FunctionTok{str\_replace}\NormalTok{(wikiscores\_data}\SpecialCharTok{$}\NormalTok{V2,}\StringTok{"\%E2\%80\%94"}\NormalTok{,}\StringTok{"{-}"}\NormalTok{)}
\NormalTok{  wikiscores\_data}\SpecialCharTok{$}\NormalTok{V2 }\OtherTok{=} \FunctionTok{sapply}\NormalTok{(wikiscores\_data}\SpecialCharTok{$}\NormalTok{V2 , }\ControlFlowTok{function}\NormalTok{(x) }\FunctionTok{URLdecode}\NormalTok{(x) )}
\NormalTok{  wikiscores\_data}\SpecialCharTok{$}\NormalTok{V2 }\OtherTok{=} \FunctionTok{str\_replace}\NormalTok{(wikiscores\_data}\SpecialCharTok{$}\NormalTok{V2,}\StringTok{"}\SpecialCharTok{\textbackslash{}\textbackslash{}}\StringTok{?}\SpecialCharTok{\textbackslash{}\textbackslash{}}\StringTok{\^{}}\SpecialCharTok{\textbackslash{}\textbackslash{}}\StringTok{\textquotesingle{}"}\NormalTok{,}\StringTok{"{-}"}\NormalTok{)}
  \CommentTok{\# covert letter grade to numeric}
\NormalTok{  wikiscores\_data}\SpecialCharTok{$}\NormalTok{V2 }\OtherTok{=} \FunctionTok{sapply}\NormalTok{(}\FunctionTok{str\_trim}\NormalTok{(wikiscores\_data}\SpecialCharTok{$}\NormalTok{V2) , lettergrade\_num)}
  \CommentTok{\# remove unit of measure (stars or discs)}
\NormalTok{  wikiscores\_data}\SpecialCharTok{$}\NormalTok{V2 }\OtherTok{=}\NormalTok{ stringr}\SpecialCharTok{::}\FunctionTok{str\_replace}\NormalTok{(wikiscores\_data}\SpecialCharTok{$}\NormalTok{V2,}\StringTok{"stars"}\NormalTok{,}\StringTok{""}\NormalTok{)}
\NormalTok{  wikiscores\_data}\SpecialCharTok{$}\NormalTok{V2 }\OtherTok{=}\NormalTok{ stringr}\SpecialCharTok{::}\FunctionTok{str\_replace}\NormalTok{(wikiscores\_data}\SpecialCharTok{$}\NormalTok{V2,}\StringTok{"discs"}\NormalTok{,}\StringTok{""}\NormalTok{)}
  \CommentTok{\# remove \%\textquotesingle{}s}
\NormalTok{  wikiscores\_data}\SpecialCharTok{$}\NormalTok{V2 }\OtherTok{=}\NormalTok{ stringr}\SpecialCharTok{::}\FunctionTok{str\_replace}\NormalTok{(wikiscores\_data}\SpecialCharTok{$}\NormalTok{V2,}\StringTok{"\%"}\NormalTok{,}\StringTok{""}\NormalTok{)}
  \CommentTok{\# remove remaining scores that contain no numbers}
\NormalTok{  wikiscores\_data }\OtherTok{=}\NormalTok{ wikiscores\_data[}\FunctionTok{grepl}\NormalTok{(}\StringTok{"}\SpecialCharTok{\textbackslash{}\textbackslash{}}\StringTok{d"}\NormalTok{,wikiscores\_data}\SpecialCharTok{$}\NormalTok{V2),]}
  
  \CommentTok{\# convert to numeric}
\NormalTok{  wikiscores\_data}\SpecialCharTok{$}\NormalTok{V2 }\OtherTok{=} \FunctionTok{sapply}\NormalTok{(wikiscores\_data}\SpecialCharTok{$}\NormalTok{V2,}
                              \ControlFlowTok{function}\NormalTok{(x) }\FunctionTok{tryCatch}\NormalTok{(}\FunctionTok{eval}\NormalTok{(}\FunctionTok{parse}\NormalTok{(}\AttributeTok{text=}\NormalTok{x)),}
                                                   \AttributeTok{error=}\ControlFlowTok{function}\NormalTok{(e)\{\},}
                                                   \AttributeTok{finally=}\ConstantTok{NULL}\NormalTok{) )}
  
  \CommentTok{\# final set of scores to put in dataset}
\NormalTok{  wikiscores\_data }\OtherTok{=}\NormalTok{ wikiscores\_data[}\SpecialCharTok{!}\FunctionTok{sapply}\NormalTok{(wikiscores\_data}\SpecialCharTok{$}\NormalTok{V2,is.null),]}
\NormalTok{  wikiscores\_data}\SpecialCharTok{$}\NormalTok{V2 }\OtherTok{=} \FunctionTok{as.numeric}\NormalTok{(wikiscores\_data}\SpecialCharTok{$}\NormalTok{V2)}
\NormalTok{  wikiscores\_data}\SpecialCharTok{$}\NormalTok{V2[wikiscores\_data}\SpecialCharTok{$}\NormalTok{V2}\SpecialCharTok{\textless{}=}\DecValTok{1}\NormalTok{] }\OtherTok{=} \DecValTok{100}\SpecialCharTok{*}\NormalTok{wikiscores\_data}\SpecialCharTok{$}\NormalTok{V2[wikiscores\_data}\SpecialCharTok{$}\NormalTok{V2}\SpecialCharTok{\textless{}=}\DecValTok{1}\NormalTok{]}
  
\NormalTok{  xpath }\OtherTok{=} \StringTok{\textquotesingle{}//*[contains(concat( " ", @class, " " ), concat( " ", "published", " " ))]\textquotesingle{}}
\NormalTok{  release\_date\_wiki }\OtherTok{=} \FunctionTok{html\_nodes}\NormalTok{(webpage, }\AttributeTok{xpath=}\NormalTok{xpath)}
\NormalTok{  release\_date\_wiki }\OtherTok{=} \FunctionTok{html\_text}\NormalTok{(release\_date\_wiki,}\AttributeTok{trim=}\NormalTok{T)}
\NormalTok{  release\_date\_wiki }\OtherTok{=}\NormalTok{ release\_date\_wiki[}\DecValTok{2}\NormalTok{]}
\end{Highlighting}
\end{Shaded}

\newpage

\hypertarget{appendix-iii-full-code-for-generating-data}{%
\section{Appendix III: Full Code for Generating
Data}\label{appendix-iii-full-code-for-generating-data}}

\begin{Shaded}
\begin{Highlighting}[]
\CommentTok{\# load packages and define helper functions}
\FunctionTok{source}\NormalTok{(}\StringTok{"Header.R"}\NormalTok{)}

\CommentTok{\# get list of artist names}
\FunctionTok{load}\NormalTok{(}\StringTok{"artistnamelist.RData"}\NormalTok{) }\CommentTok{\# loads object "artistnames"}

\CommentTok{\# For each name in the list, create data of albums and reviews}
\NormalTok{artistdata }\OtherTok{=} \ConstantTok{NULL}
\ControlFlowTok{for}\NormalTok{ (aname }\ControlFlowTok{in}\NormalTok{ artistnames[}\DecValTok{694}\SpecialCharTok{:}\DecValTok{1000}\NormalTok{]) \{}

\DocumentationTok{\#\# SPOTIFY DATA \#\#\#\#\#\#\#\#\#\#\#\#\#\#\#\#\#\#\#\#\#\#\#\#\#\#\#\#\#\#\#\#\#\#\#\#\#}

\NormalTok{  id }\OtherTok{=} \StringTok{\textquotesingle{}3f766570eef74a6986a646bf7dd73102\textquotesingle{}}
\NormalTok{  secret }\OtherTok{=} \StringTok{\textquotesingle{}5f884eb17551445ba76ab797ec7505b6\textquotesingle{}}
  \FunctionTok{Sys.setenv}\NormalTok{(}\AttributeTok{SPOTIFY\_CLIENT\_ID =}\NormalTok{ id)}
  \FunctionTok{Sys.setenv}\NormalTok{(}\AttributeTok{SPOTIFY\_CLIENT\_SECRET =}\NormalTok{ secret)}
\NormalTok{  access\_token }\OtherTok{=}\NormalTok{ spotifyr}\SpecialCharTok{::}\FunctionTok{get\_spotify\_access\_token}\NormalTok{()}

  \CommentTok{\# artist info}
\NormalTok{  search }\OtherTok{=}\NormalTok{ spotifyr}\SpecialCharTok{::}\FunctionTok{search\_spotify}\NormalTok{(}\AttributeTok{q=}\NormalTok{aname, }\AttributeTok{type=}\StringTok{"artist"}\NormalTok{)}
\NormalTok{  artist\_id }\OtherTok{=}\NormalTok{ search}\SpecialCharTok{$}\NormalTok{id[}\DecValTok{1}\NormalTok{]}
\NormalTok{  artist }\OtherTok{=}\NormalTok{ spotifyr}\SpecialCharTok{::}\FunctionTok{get\_artist}\NormalTok{(}\AttributeTok{id=}\NormalTok{artist\_id, }\AttributeTok{authorization=}\NormalTok{access\_token)}
\NormalTok{  artist\_name }\OtherTok{=}\NormalTok{ artist}\SpecialCharTok{$}\NormalTok{name}

  \CommentTok{\# album list}
\NormalTok{  albums }\OtherTok{=}\NormalTok{ spotifyr}\SpecialCharTok{::}\FunctionTok{get\_artist\_albums}\NormalTok{(}\AttributeTok{id=}\NormalTok{artist\_id,}\AttributeTok{include\_groups=}\StringTok{"album"}\NormalTok{,}\AttributeTok{market=}\StringTok{"CA"}\NormalTok{,}\AttributeTok{limit=}\DecValTok{50}\NormalTok{,}\AttributeTok{authorization=}\NormalTok{access\_token)}
\NormalTok{  nalb }\OtherTok{=} \FunctionTok{length}\NormalTok{(albums)}
\NormalTok{  offset }\OtherTok{=} \DecValTok{1}
  \ControlFlowTok{while}\NormalTok{(nalb}\SpecialCharTok{\textgreater{}}\DecValTok{0}\NormalTok{) \{}
\NormalTok{      offset }\OtherTok{=}\NormalTok{ offset }\SpecialCharTok{+} \DecValTok{50}
\NormalTok{      newalbs }\OtherTok{=}\NormalTok{ spotifyr}\SpecialCharTok{::}\FunctionTok{get\_artist\_albums}\NormalTok{(}\AttributeTok{id=}\NormalTok{artist\_id,}\AttributeTok{include\_groups=}\StringTok{"album"}\NormalTok{,}\AttributeTok{market=}\StringTok{"CA"}\NormalTok{,}\AttributeTok{limit=}\DecValTok{50}\NormalTok{,}\AttributeTok{offset=}\NormalTok{offset,}\AttributeTok{authorization=}\NormalTok{access\_token)}
\NormalTok{      nalb }\OtherTok{=} \FunctionTok{length}\NormalTok{(newalbs)}
\NormalTok{      albums }\OtherTok{=} \FunctionTok{rbind}\NormalTok{(albums,newalbs)}
\NormalTok{  \}}
  
  
  \CommentTok{\# Dedupe album list}
\NormalTok{  albums}\SpecialCharTok{$}\NormalTok{name }\OtherTok{=} \FunctionTok{gsub}\NormalTok{(}\StringTok{"}\SpecialCharTok{\textbackslash{}\textbackslash{}}\StringTok{s*}\SpecialCharTok{\textbackslash{}\textbackslash{}}\StringTok{([\^{}}\SpecialCharTok{\textbackslash{}\textbackslash{}}\StringTok{)]+}\SpecialCharTok{\textbackslash{}\textbackslash{}}\StringTok{)"}\NormalTok{,}\StringTok{""}\NormalTok{,albums}\SpecialCharTok{$}\NormalTok{name) }\CommentTok{\# remove "deluxe" versions, etc.}
  \ControlFlowTok{if}\NormalTok{(}\FunctionTok{length}\NormalTok{(albums}\SpecialCharTok{$}\NormalTok{name)}\SpecialCharTok{\textgreater{}}\DecValTok{0}\NormalTok{) \{ }\CommentTok{\# only proceed if they have albums!}
\NormalTok{    albums }\OtherTok{=}\NormalTok{ albums[}\SpecialCharTok{!}\FunctionTok{duplicated}\NormalTok{(albums}\SpecialCharTok{$}\NormalTok{name),]}
\NormalTok{    albums }\OtherTok{=}\NormalTok{ albums[}\SpecialCharTok{!}\FunctionTok{duplicated}\NormalTok{(albums}\SpecialCharTok{$}\NormalTok{release\_date),]}
\NormalTok{  \} }\ControlFlowTok{else}\NormalTok{ \{}
    \FunctionTok{print}\NormalTok{(}\FunctionTok{paste0}\NormalTok{(}\StringTok{"Skipping "}\NormalTok{, artist\_name, }\StringTok{"{-}{-}{-}{-}no albums!"}\NormalTok{))}
\NormalTok{  \}}

  \CommentTok{\# loop through albums}
\NormalTok{  album\_ids }\OtherTok{=}\NormalTok{ albums}\SpecialCharTok{$}\NormalTok{id}
  \ControlFlowTok{for}\NormalTok{ (id }\ControlFlowTok{in}\NormalTok{ album\_ids) \{}
\NormalTok{    example\_album}\OtherTok{=}\NormalTok{spotifyr}\SpecialCharTok{::}\FunctionTok{get\_album}\NormalTok{(}\AttributeTok{id=}\NormalTok{id,}\AttributeTok{authorization=}\NormalTok{access\_token)}

    \DocumentationTok{\#\# LAST FM \#\#\#\#\#\#\#\#\#\#\#\#\#\#\#\#\#\#\#\#\#\#\#\#\#\#\#\#\#\#\#\#\#\#\#\#\#\#\#\#\#\#\#\#\#\#\#\#\#\#\#\#\#\#\#\#\#\#\#\#\#\#}
\NormalTok{    api\_key }\OtherTok{=} \StringTok{"749b4800592668088e8a563845cf906f"}
    \CommentTok{\# \textquotesingle{}\&\textquotesingle{} and \textquotesingle{}+\textquotesingle{} and \textquotesingle{}\#\textquotesingle{} cause issues}
\NormalTok{    url\_artist }\OtherTok{=}\NormalTok{ stringr}\SpecialCharTok{::}\FunctionTok{str\_replace\_all}\NormalTok{(artist\_name,}\StringTok{"}\SpecialCharTok{\textbackslash{}\textbackslash{}}\StringTok{+"}\NormalTok{,}\StringTok{"\%2B"}\NormalTok{)}
\NormalTok{    url\_album }\OtherTok{=}\NormalTok{ stringr}\SpecialCharTok{::}\FunctionTok{str\_replace\_all}\NormalTok{(example\_album}\SpecialCharTok{$}\NormalTok{name,}\StringTok{"}\SpecialCharTok{\textbackslash{}\textbackslash{}}\StringTok{+"}\NormalTok{,}\StringTok{"\%2B"}\NormalTok{)}
\NormalTok{    url\_artist }\OtherTok{=}\NormalTok{ stringr}\SpecialCharTok{::}\FunctionTok{str\_replace\_all}\NormalTok{(url\_artist,}\StringTok{"\&"}\NormalTok{,}\StringTok{"\%26"}\NormalTok{)}
\NormalTok{    url\_album }\OtherTok{=}\NormalTok{ stringr}\SpecialCharTok{::}\FunctionTok{str\_replace\_all}\NormalTok{(url\_album,}\StringTok{"\&"}\NormalTok{,}\StringTok{"\%26"}\NormalTok{)}
\NormalTok{    url\_artist }\OtherTok{=}\NormalTok{ stringr}\SpecialCharTok{::}\FunctionTok{str\_replace\_all}\NormalTok{(url\_artist,}\StringTok{"\#"}\NormalTok{,}\StringTok{"\%23"}\NormalTok{)}
\NormalTok{    url\_album }\OtherTok{=}\NormalTok{ stringr}\SpecialCharTok{::}\FunctionTok{str\_replace\_all}\NormalTok{(url\_album,}\StringTok{"\#"}\NormalTok{,}\StringTok{"\%23"}\NormalTok{)}

\NormalTok{    url }\OtherTok{=} \FunctionTok{paste0}\NormalTok{(}\StringTok{"http://ws.audioscrobbler.com/2.0/"}\NormalTok{,}
                 \StringTok{"?method=album.getinfo"}\NormalTok{,}
                 \StringTok{"\&api\_key="}\NormalTok{,api\_key,}
                 \StringTok{"\&artist="}\NormalTok{, }\FunctionTok{gsub}\NormalTok{(}\StringTok{" "}\NormalTok{, }\StringTok{"+"}\NormalTok{, url\_artist),}
                 \StringTok{"\&album="}\NormalTok{, }\FunctionTok{gsub}\NormalTok{(}\StringTok{" "}\NormalTok{, }\StringTok{"+"}\NormalTok{, url\_album),}
                 \StringTok{"\&format=json"}\NormalTok{)}
\NormalTok{    data\_json }\OtherTok{=}\NormalTok{ httr}\SpecialCharTok{::}\FunctionTok{GET}\NormalTok{(url)}
\NormalTok{    data\_json }\OtherTok{=}\NormalTok{ jsonlite}\SpecialCharTok{::}\FunctionTok{fromJSON}\NormalTok{(}\FunctionTok{rawToChar}\NormalTok{(data\_json}\SpecialCharTok{$}\NormalTok{content))}
\NormalTok{    album\_scrobbles }\OtherTok{=} \FunctionTok{as.integer}\NormalTok{(data\_json}\SpecialCharTok{$}\NormalTok{album}\SpecialCharTok{$}\NormalTok{playcount)}
    \ControlFlowTok{if}\NormalTok{ (}\FunctionTok{length}\NormalTok{(album\_scrobbles)}\SpecialCharTok{==}\DecValTok{0}\NormalTok{) \{ }\CommentTok{\# handle missing scrobbles}
\NormalTok{      album\_scrobbles }\OtherTok{=} \ConstantTok{NA}
\NormalTok{    \}}
    \DocumentationTok{\#\#\#\#\#\#\#\#\#\#\#\#\#\#\#\#\#\#\#\#\#\#\#\#\#\#\#\#\#\#\#\#\#\#\#\#\#\#\#\#\#\#\#\#\#\#\#\#\#\#\#\#\#\#\#\#\#\#\#\#\#\#\#\#\#\#\#\#\#\#\#\#\#}

    \DocumentationTok{\#\# CRITIC SCORES FROM WIKIPEDIA \#\#\#\#\#\#\#\#\#\#\#\#\#\#\#\#\#\#\#\#\#\#\#\#\#\#\#\#\#\#\#\#\#\#\#\#\#\#\#\#\#}

    \CommentTok{\# Use google api to search for the (most likely) correct wikipedia page}
    \CommentTok{\# (note that this is a custom search api that searches only wikipedia)}
\NormalTok{     google.key }\OtherTok{=} \StringTok{\textquotesingle{}AIzaSyBKJ9b9c1zKEEUs{-}g2PTrosIHO8epctAls\textquotesingle{}}
\NormalTok{     google.cx }\OtherTok{=} \StringTok{\textquotesingle{}2fb4b305ff06c300d\textquotesingle{}}
     
     \CommentTok{\# keywords to be searched in google, separated by \textquotesingle{}+\textquotesingle{}}
\NormalTok{     keyword }\OtherTok{=} \FunctionTok{paste0}\NormalTok{(artist\_name,}\StringTok{" "}\NormalTok{,example\_album}\SpecialCharTok{$}\NormalTok{name,}\StringTok{" "}\NormalTok{,}\StringTok{"album"}\NormalTok{)}
\NormalTok{     keyword }\OtherTok{=} \FunctionTok{gsub}\NormalTok{(}\StringTok{"[[:punct:]]"}\NormalTok{, }\StringTok{" "}\NormalTok{, keyword)}
\NormalTok{     keyword }\OtherTok{=}\NormalTok{ stringr}\SpecialCharTok{::}\FunctionTok{str\_squish}\NormalTok{(keyword)}
\NormalTok{     keyword }\OtherTok{=} \FunctionTok{gsub}\NormalTok{(}\StringTok{" "}\NormalTok{, }\StringTok{"+"}\NormalTok{, keyword)}
    
     \CommentTok{\# api call}
\NormalTok{     url }\OtherTok{=} \FunctionTok{paste0}\NormalTok{(}\StringTok{"https://www.googleapis.com/customsearch/v1?"}
\NormalTok{                  , }\StringTok{"key="}\NormalTok{, google.key}
\NormalTok{                  , }\StringTok{"\&q="}\NormalTok{, keyword}
\NormalTok{                  , }\StringTok{"\&gl=us"}         
\NormalTok{                  , }\StringTok{"\&hl=en"}                
\NormalTok{                  , }\StringTok{"\&cx="}\NormalTok{, google.cx}
\NormalTok{                  , }\StringTok{"\&fields=items(link)"}
\NormalTok{     )}
\NormalTok{     googlesearch }\OtherTok{=}\NormalTok{ httr}\SpecialCharTok{::}\FunctionTok{GET}\NormalTok{(url)}
\NormalTok{     content }\OtherTok{=} \FunctionTok{rawToChar}\NormalTok{(googlesearch}\SpecialCharTok{$}\NormalTok{content); }\FunctionTok{Encoding}\NormalTok{(content) }\OtherTok{=} \StringTok{"UTF{-}8"}
\NormalTok{     searchresults }\OtherTok{=}\NormalTok{ jsonlite}\SpecialCharTok{::}\FunctionTok{fromJSON}\NormalTok{(content)}
     
     \ControlFlowTok{if}\NormalTok{ (}\FunctionTok{length}\NormalTok{(searchresults)}\SpecialCharTok{!=}\DecValTok{0}\NormalTok{) \{ }\CommentTok{\# only proceed if google search finds at least one url}
     
\NormalTok{       wikiurl }\OtherTok{=}\NormalTok{ searchresults}\SpecialCharTok{$}\NormalTok{items}\SpecialCharTok{$}\NormalTok{link[}\DecValTok{1}\NormalTok{]}
  
      \CommentTok{\# Wikipedia API {-}\textgreater{} using the page title, parse the album page and extract album scores}
        \CommentTok{\# do some cleaning to the url to get the proper page name}
          \CommentTok{\# deal with special characters}
\NormalTok{          pgname }\OtherTok{=}\NormalTok{ curl}\SpecialCharTok{::}\FunctionTok{curl\_unescape}\NormalTok{(wikiurl)}
          \CommentTok{\# keep the end of the url {-}\textgreater{} the page name wikipedia api wants}
\NormalTok{          pgname }\OtherTok{=} \FunctionTok{gsub}\NormalTok{(}\StringTok{"https://.*}\SpecialCharTok{\textbackslash{}\textbackslash{}}\StringTok{.wikipedia}\SpecialCharTok{\textbackslash{}\textbackslash{}}\StringTok{.org.*/"}\NormalTok{,}\StringTok{""}\NormalTok{,pgname)}
        
        \CommentTok{\# get the language of the wiki article}
\NormalTok{          wikilang }\OtherTok{=} \FunctionTok{gsub}\NormalTok{(}\StringTok{"https://"}\NormalTok{,}\StringTok{""}\NormalTok{,wikiurl)}
\NormalTok{          wikilang }\OtherTok{=} \FunctionTok{gsub}\NormalTok{(}\StringTok{"}\SpecialCharTok{\textbackslash{}\textbackslash{}}\StringTok{.wikipedia.org.*"}\NormalTok{,}\StringTok{""}\NormalTok{,wikilang)}
          
        \CommentTok{\# obtain page content}
          \CommentTok{\# try english language first, otherwise use language of url}
\NormalTok{          error }\OtherTok{=} \DecValTok{0}
          \FunctionTok{tryCatch}\NormalTok{(}
              \AttributeTok{expr    =}\NormalTok{ \{webpage }\OtherTok{=}\NormalTok{ WikipediR}\SpecialCharTok{::}\FunctionTok{page\_content}\NormalTok{(}\AttributeTok{language=}\StringTok{"en"}\NormalTok{,}\AttributeTok{project=}\StringTok{"wikipedia"}\NormalTok{,}\AttributeTok{page\_name=}\NormalTok{pgname)\}}
\NormalTok{            , }\AttributeTok{error   =} \ControlFlowTok{function}\NormalTok{(e)\{\}}
\NormalTok{            , }\AttributeTok{finally =}\NormalTok{ \{}\FunctionTok{tryCatch}\NormalTok{(}
                    \AttributeTok{expr =}\NormalTok{ \{webpage }\OtherTok{=}\NormalTok{ WikipediR}\SpecialCharTok{::}\FunctionTok{page\_content}\NormalTok{(}\AttributeTok{language=}\NormalTok{wikilang,}\AttributeTok{project=}\StringTok{"wikipedia"}\NormalTok{,}\AttributeTok{page\_name=}\NormalTok{pgname)\}}
\NormalTok{                  , }\AttributeTok{error =} \ControlFlowTok{function}\NormalTok{(e)\{}\FunctionTok{print}\NormalTok{(}\FunctionTok{paste0}\NormalTok{(}\StringTok{"Skipping url: "}\NormalTok{,wikiurl)); }\FunctionTok{assign}\NormalTok{(}\StringTok{"error"}\NormalTok{,}\DecValTok{1}\NormalTok{,}\AttributeTok{envir =}\NormalTok{ .GlobalEnv)\}}
\NormalTok{                  , }\AttributeTok{finally =}\NormalTok{ \{\})\})}
        
        \ControlFlowTok{if}\NormalTok{ (error}\SpecialCharTok{!=}\DecValTok{1}\NormalTok{) \{ }\CommentTok{\# only proceed if wikipedia url worked}
        \CommentTok{\# parse wikipedia content}
\NormalTok{        webpage }\OtherTok{=}\NormalTok{ rvest}\SpecialCharTok{::}\FunctionTok{read\_html}\NormalTok{(webpage}\SpecialCharTok{$}\NormalTok{parse}\SpecialCharTok{$}\NormalTok{text}\SpecialCharTok{$}\StringTok{\textasciigrave{}}\AttributeTok{*}\StringTok{\textasciigrave{}}\NormalTok{)}
\NormalTok{        wikiscores }\OtherTok{=}\NormalTok{ rvest}\SpecialCharTok{::}\FunctionTok{html\_nodes}\NormalTok{(webpage, }\AttributeTok{xpath=}\StringTok{\textquotesingle{}//*[contains(concat( " ", @class, " " ), concat( " ", "floatright", " " ))]//td\textquotesingle{}}\NormalTok{)}
\NormalTok{        wikiscores\_stars }\OtherTok{=}\NormalTok{ rvest}\SpecialCharTok{::}\FunctionTok{html\_attr}\NormalTok{(rvest}\SpecialCharTok{::}\FunctionTok{html\_element}\NormalTok{(wikiscores,}\AttributeTok{css=}\StringTok{\textquotesingle{}span\textquotesingle{}}\NormalTok{),}\StringTok{\textquotesingle{}title\textquotesingle{}}\NormalTok{)}
\NormalTok{        wikiscores }\OtherTok{=}\NormalTok{ rvest}\SpecialCharTok{::}\FunctionTok{html\_text}\NormalTok{(wikiscores,}\AttributeTok{trim=}\NormalTok{T)}
  
        \CommentTok{\# some formatting for the scores {-}\textgreater{} convert them all into standard numeric out of 100}
\NormalTok{        wikiscores\_data }\OtherTok{=} \FunctionTok{as.data.frame}\NormalTok{(}\FunctionTok{matrix}\NormalTok{(wikiscores,}\AttributeTok{ncol=}\DecValTok{2}\NormalTok{,}\AttributeTok{byrow=}\NormalTok{T))}
\NormalTok{        wikiscores\_stars\_data }\OtherTok{=} \FunctionTok{as.data.frame}\NormalTok{(}\FunctionTok{matrix}\NormalTok{(wikiscores\_stars,}\AttributeTok{ncol=}\DecValTok{2}\NormalTok{,}\AttributeTok{byrow=}\NormalTok{T))}
          \CommentTok{\# deal with specific review websites}
\NormalTok{           selector }\OtherTok{=}\NormalTok{ (}\SpecialCharTok{!}\FunctionTok{is.na}\NormalTok{(wikiscores\_stars\_data}\SpecialCharTok{$}\NormalTok{V2) }\SpecialCharTok{\&} \SpecialCharTok{!}\NormalTok{(}\FunctionTok{tolower}\NormalTok{(wikiscores\_data}\SpecialCharTok{$}\NormalTok{V1)}\SpecialCharTok{==}\StringTok{"tom hull – on the Web"}\NormalTok{))}
\NormalTok{           wikiscores\_data}\SpecialCharTok{$}\NormalTok{V2[selector] }\OtherTok{=}\NormalTok{ wikiscores\_stars\_data}\SpecialCharTok{$}\NormalTok{V2[selector]}
\NormalTok{           wikiscores\_data }\OtherTok{=}\NormalTok{ wikiscores\_data[wikiscores\_data}\SpecialCharTok{$}\NormalTok{V1}\SpecialCharTok{!=}\StringTok{"Metacritic"}\NormalTok{,] }\CommentTok{\# this is a review aggregator}
\NormalTok{           wikiscores\_data }\OtherTok{=}\NormalTok{ wikiscores\_data[wikiscores\_data}\SpecialCharTok{$}\NormalTok{V1}\SpecialCharTok{!=}\StringTok{"AnyDecentMusic?"}\NormalTok{,] }\CommentTok{\# this is a review aggregator}
          \CommentTok{\# remove [] and content inside}
\NormalTok{           wikiscores\_data}\SpecialCharTok{$}\NormalTok{V2 }\OtherTok{=} \FunctionTok{gsub}\NormalTok{(}\StringTok{"}\SpecialCharTok{\textbackslash{}\textbackslash{}}\StringTok{[.*?}\SpecialCharTok{\textbackslash{}\textbackslash{}}\StringTok{]"}\NormalTok{,}\StringTok{""}\NormalTok{,wikiscores\_data}\SpecialCharTok{$}\NormalTok{V2)}
          \CommentTok{\# remove () but NOT CONTENT (sometimes the score is contained within ()}
\NormalTok{           wikiscores\_data}\SpecialCharTok{$}\NormalTok{V2 }\OtherTok{=}\NormalTok{ stringr}\SpecialCharTok{::}\FunctionTok{str\_replace}\NormalTok{(wikiscores\_data}\SpecialCharTok{$}\NormalTok{V2,}\StringTok{"}\SpecialCharTok{\textbackslash{}\textbackslash{}}\StringTok{("}\NormalTok{,}\StringTok{""}\NormalTok{)}
\NormalTok{           wikiscores\_data}\SpecialCharTok{$}\NormalTok{V2 }\OtherTok{=}\NormalTok{ stringr}\SpecialCharTok{::}\FunctionTok{str\_replace}\NormalTok{(wikiscores\_data}\SpecialCharTok{$}\NormalTok{V2,}\StringTok{"}\SpecialCharTok{\textbackslash{}\textbackslash{}}\StringTok{)"}\NormalTok{,}\StringTok{""}\NormalTok{)}
          \CommentTok{\# fix ill{-}behaved special characters}
\NormalTok{           wikiscores\_data}\SpecialCharTok{$}\NormalTok{V2 }\OtherTok{=} \FunctionTok{sapply}\NormalTok{(wikiscores\_data}\SpecialCharTok{$}\NormalTok{V2 , }\ControlFlowTok{function}\NormalTok{(x) utils}\SpecialCharTok{::}\FunctionTok{URLencode}\NormalTok{(x) )}
\NormalTok{           wikiscores\_data}\SpecialCharTok{$}\NormalTok{V2 }\OtherTok{=}\NormalTok{ stringr}\SpecialCharTok{::}\FunctionTok{str\_replace}\NormalTok{(wikiscores\_data}\SpecialCharTok{$}\NormalTok{V2,}\StringTok{"\%E2\%88\%92"}\NormalTok{,}\StringTok{"{-}"}\NormalTok{)}
\NormalTok{           wikiscores\_data}\SpecialCharTok{$}\NormalTok{V2 }\OtherTok{=}\NormalTok{ stringr}\SpecialCharTok{::}\FunctionTok{str\_replace}\NormalTok{(wikiscores\_data}\SpecialCharTok{$}\NormalTok{V2,}\StringTok{"\%E2\%80\%93"}\NormalTok{,}\StringTok{"{-}"}\NormalTok{)}
\NormalTok{           wikiscores\_data}\SpecialCharTok{$}\NormalTok{V2 }\OtherTok{=}\NormalTok{ stringr}\SpecialCharTok{::}\FunctionTok{str\_replace}\NormalTok{(wikiscores\_data}\SpecialCharTok{$}\NormalTok{V2,}\StringTok{"\%E2\%80\%94"}\NormalTok{,}\StringTok{"{-}"}\NormalTok{)}
\NormalTok{           wikiscores\_data}\SpecialCharTok{$}\NormalTok{V2 }\OtherTok{=} \FunctionTok{sapply}\NormalTok{(wikiscores\_data}\SpecialCharTok{$}\NormalTok{V2 , }\ControlFlowTok{function}\NormalTok{(x) utils}\SpecialCharTok{::}\FunctionTok{URLdecode}\NormalTok{(x) )}
\NormalTok{           wikiscores\_data}\SpecialCharTok{$}\NormalTok{V2 }\OtherTok{=}\NormalTok{ stringr}\SpecialCharTok{::}\FunctionTok{str\_replace}\NormalTok{(wikiscores\_data}\SpecialCharTok{$}\NormalTok{V2,}\StringTok{"}\SpecialCharTok{\textbackslash{}\textbackslash{}}\StringTok{?}\SpecialCharTok{\textbackslash{}\textbackslash{}}\StringTok{\^{}}\SpecialCharTok{\textbackslash{}\textbackslash{}}\StringTok{\textquotesingle{}"}\NormalTok{,}\StringTok{"{-}"}\NormalTok{)}
        \CommentTok{\# covert letter grade to numeric}
\NormalTok{          wikiscores\_data}\SpecialCharTok{$}\NormalTok{V2 }\OtherTok{=} \FunctionTok{sapply}\NormalTok{(stringr}\SpecialCharTok{::}\FunctionTok{str\_trim}\NormalTok{(wikiscores\_data}\SpecialCharTok{$}\NormalTok{V2) , lettergrade\_num)}
          \CommentTok{\# remove unit of measure (stars or discs)}
\NormalTok{           wikiscores\_data}\SpecialCharTok{$}\NormalTok{V2 }\OtherTok{=}\NormalTok{ stringr}\SpecialCharTok{::}\FunctionTok{str\_replace}\NormalTok{(wikiscores\_data}\SpecialCharTok{$}\NormalTok{V2,}\StringTok{"stars"}\NormalTok{,}\StringTok{""}\NormalTok{)}
\NormalTok{           wikiscores\_data}\SpecialCharTok{$}\NormalTok{V2 }\OtherTok{=}\NormalTok{ stringr}\SpecialCharTok{::}\FunctionTok{str\_replace}\NormalTok{(wikiscores\_data}\SpecialCharTok{$}\NormalTok{V2,}\StringTok{"discs"}\NormalTok{,}\StringTok{""}\NormalTok{)}
          \CommentTok{\# remove \%\textquotesingle{}s}
\NormalTok{           wikiscores\_data}\SpecialCharTok{$}\NormalTok{V2 }\OtherTok{=}\NormalTok{ stringr}\SpecialCharTok{::}\FunctionTok{str\_replace}\NormalTok{(wikiscores\_data}\SpecialCharTok{$}\NormalTok{V2,}\StringTok{"\%"}\NormalTok{,}\StringTok{""}\NormalTok{)}
          \CommentTok{\# remove remaining scores that contain no numbers}
\NormalTok{           wikiscores\_data }\OtherTok{=}\NormalTok{ wikiscores\_data[}\FunctionTok{grepl}\NormalTok{(}\StringTok{"}\SpecialCharTok{\textbackslash{}\textbackslash{}}\StringTok{d"}\NormalTok{,wikiscores\_data}\SpecialCharTok{$}\NormalTok{V2),]}
  
          \CommentTok{\# convert to numeric}
\NormalTok{          wikiscores\_data}\SpecialCharTok{$}\NormalTok{V2 }\OtherTok{=} \FunctionTok{sapply}\NormalTok{(wikiscores\_data}\SpecialCharTok{$}\NormalTok{V2,}
                                      \ControlFlowTok{function}\NormalTok{(x) }\FunctionTok{tryCatch}\NormalTok{(}\FunctionTok{eval}\NormalTok{(}\FunctionTok{parse}\NormalTok{(}\AttributeTok{text=}\NormalTok{x)),}
                                                           \AttributeTok{error=}\ControlFlowTok{function}\NormalTok{(e)\{\},}
                                                           \AttributeTok{finally=}\ConstantTok{NULL}\NormalTok{) )}
        \CommentTok{\# final set of scores to put in dataset}
\NormalTok{          wikiscores\_data }\OtherTok{=}\NormalTok{ wikiscores\_data[}\SpecialCharTok{!}\FunctionTok{sapply}\NormalTok{(wikiscores\_data}\SpecialCharTok{$}\NormalTok{V2,is.null),]}
\NormalTok{          wikiscores\_data}\SpecialCharTok{$}\NormalTok{V2 }\OtherTok{=} \FunctionTok{as.numeric}\NormalTok{(wikiscores\_data}\SpecialCharTok{$}\NormalTok{V2)}
\NormalTok{          wikiscores\_data}\SpecialCharTok{$}\NormalTok{V2[wikiscores\_data}\SpecialCharTok{$}\NormalTok{V2}\SpecialCharTok{\textless{}=}\DecValTok{1}\NormalTok{] }\OtherTok{=} \DecValTok{100}\SpecialCharTok{*}\NormalTok{wikiscores\_data}\SpecialCharTok{$}\NormalTok{V2[wikiscores\_data}\SpecialCharTok{$}\NormalTok{V2}\SpecialCharTok{\textless{}=}\DecValTok{1}\NormalTok{]}
      \DocumentationTok{\#\#\#\#\#\#\#\#\#\#\#\#\#\#\#\#\#\#\#\#\#\#\#\#\#\#\#\#\#\#\#\#\#\#\#\#\#\#\#\#\#\#\#\#\#\#\#\#\#\#\#\#\#\#\#\#\#\#\#\#\#\#\#\#\#\#\#\#\#\#\#\#\#}
  
      \DocumentationTok{\#\# CRITIC SCORES FROM WIKIPEDIA \#\#\#\#\#\#\#\#\#\#\#\#\#\#\#\#\#\#\#\#\#\#\#\#\#\#\#\#\#\#\#\#\#\#\#\#\#\#\#\#\#}
\NormalTok{        release\_date\_wiki }\OtherTok{=}\NormalTok{ rvest}\SpecialCharTok{::}\FunctionTok{html\_nodes}\NormalTok{(webpage, }\AttributeTok{xpath=}\StringTok{\textquotesingle{}//*[contains(concat( " ", @class, " " ), concat( " ", "published", " " ))]\textquotesingle{}}\NormalTok{)}
\NormalTok{        release\_date\_wiki }\OtherTok{=}\NormalTok{ rvest}\SpecialCharTok{::}\FunctionTok{html\_text}\NormalTok{(release\_date\_wiki,}\AttributeTok{trim=}\NormalTok{T)}
\NormalTok{        release\_date\_wiki }\OtherTok{=}\NormalTok{ release\_date\_wiki[}\DecValTok{2}\NormalTok{]}
      \DocumentationTok{\#\#\#\#\#\#\#\#\#\#\#\#\#\#\#\#\#\#\#\#\#\#\#\#\#\#\#\#\#\#\#\#\#\#\#\#\#\#\#\#\#\#\#\#\#\#\#\#\#\#\#\#\#\#\#\#\#\#\#\#\#\#\#\#\#\#\#\#\#\#\#\#\#}
      
\NormalTok{      wiki\_scores }\OtherTok{=} \FunctionTok{list}\NormalTok{(wikiscores\_data}\SpecialCharTok{$}\NormalTok{V2)}
        
\NormalTok{    \} }\ControlFlowTok{else}\NormalTok{ \{ }\CommentTok{\# if no wiki article found, set variables to empty }
\NormalTok{      release\_date\_wiki }\OtherTok{=} \ConstantTok{NA}
\NormalTok{      wiki\_scores }\OtherTok{=} \FunctionTok{list}\NormalTok{(}\FunctionTok{numeric}\NormalTok{(}\DecValTok{0}\NormalTok{))}
\NormalTok{    \}}
     
     
    \CommentTok{\# grow the dataset with the additional tracks and album review info}
\NormalTok{      artistdata }\OtherTok{=} \FunctionTok{rbind}\NormalTok{(artistdata,}
                  \FunctionTok{cbind}\NormalTok{(artist\_name,}
\NormalTok{                        artist\_id,}
\NormalTok{                        artist}\SpecialCharTok{$}\NormalTok{followers}\SpecialCharTok{$}\NormalTok{total,}
\NormalTok{                        example\_album}\SpecialCharTok{$}\NormalTok{name,}
\NormalTok{                        id,}
\NormalTok{                        example\_album}\SpecialCharTok{$}\NormalTok{popularity,}
\NormalTok{                        example\_album}\SpecialCharTok{$}\NormalTok{release\_date,}
\NormalTok{                        album\_scrobbles,}
\NormalTok{                        release\_date\_wiki,}
\NormalTok{                        wiki\_scores))}
      
    \FunctionTok{print}\NormalTok{(}\FunctionTok{cbind}\NormalTok{(artist\_name,}
\NormalTok{                artist\_id,}
\NormalTok{                artist}\SpecialCharTok{$}\NormalTok{followers}\SpecialCharTok{$}\NormalTok{total,}
\NormalTok{                example\_album}\SpecialCharTok{$}\NormalTok{name,}
\NormalTok{                id,}
\NormalTok{                example\_album}\SpecialCharTok{$}\NormalTok{popularity,}
\NormalTok{                example\_album}\SpecialCharTok{$}\NormalTok{release\_date,}
\NormalTok{                album\_scrobbles,}
\NormalTok{                release\_date\_wiki,}
\NormalTok{                wiki\_scores))}
\NormalTok{    \} }\CommentTok{\# end of if(error!=1)}
\NormalTok{    \} }\CommentTok{\# end of album loop}
\NormalTok{\} }\CommentTok{\# end of artist loop}
\FunctionTok{colnames}\NormalTok{(artistdata) }\OtherTok{=} \FunctionTok{c}\NormalTok{(}\StringTok{"artist\_name"}\NormalTok{,}\StringTok{"artist\_id"}\NormalTok{,}\StringTok{"artist\_followers"}\NormalTok{,}\StringTok{"album\_name"}\NormalTok{,}\StringTok{"album\_id"}\NormalTok{,}
                         \StringTok{"album\_popularity"}\NormalTok{, }\StringTok{"album\_releasedate"}\NormalTok{,}\StringTok{"album\_scrobbles"}\NormalTok{,}\StringTok{"release\_date\_wiki"}\NormalTok{,}\StringTok{"wiki\_scores"}\NormalTok{)}
\NormalTok{artistdata }\OtherTok{=} \FunctionTok{as.data.frame}\NormalTok{(artistdata)}
\DocumentationTok{\#\#\#\#\#\#\#\#\#\#\#\#\#\#\#\#\#\#\#\#\#\#\#\#\#\#\#\#\#\#\#\#\#\#\#\#\#\#\#\#\#\#\#\#\#\#\#\#\#\#\#\#\#\#\#\#\#}
\end{Highlighting}
\end{Shaded}


  \bibliography{references.bib}


% should be about here.
% We'll put doublespacing end here

\end{document}
